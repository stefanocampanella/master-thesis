%----------------------------------------------------------------------------------------
% BEGINNING OF THE FRONTPAGE
%----------------------------------------------------------------------------------------

\thispagestyle{empty}

\begingroup
\center % Center everything on the page

%----------------------------------------------------------------------------------------
%	HEADING SECTIONS
%----------------------------------------------------------------------------------------

\textsc{\LARGE Università degli Studi di Bari}\\[1.5cm]
\textsc{\Large Dipartimento Interateneo di Fisica ``M. Merlin''}\\[0.5cm]
\textsc{\large Corso di Laurea Magistrale in Fisica}\\[0.5cm]
\textsc{\large Tesi di Laurea in Fisica Teorica}\\[0.5cm]

%----------------------------------------------------------------------------------------
%	TITLE SECTION
%----------------------------------------------------------------------------------------

\HRule \\[0.4cm]
{ \huge 
  \bfseries Spectroscopy of Charmed Hadrons: \\ Facing the Latest Experimental Results \\ with the Theory \par
}
\HRule \\[1.5cm]

%----------------------------------------------------------------------------------------
%	AUTHOR SECTION
%----------------------------------------------------------------------------------------

\begin{minipage}{0.4\textwidth}
\begin{flushleft} \large
\emph{Relatore:} \\
Dott.ssa Fulvia \textsc{De Fazio}\\
\end{flushleft}
\end{minipage}
~
\begin{minipage}{0.4\textwidth}
\begin{flushright} \large
\emph{Laureando:}\\
Stefano \textsc{Campanella}
\end{flushright}
\end{minipage}\\[4cm]

\null
\vfill

%----------------------------------------------------------------------------------------
%	DATE SECTION
%----------------------------------------------------------------------------------------

{\large \textsc{Anno Accademico 2016/2017} }

\endgroup

%----------------------------------------------------------------------------------------
% END OF THE FRONTPAGE
%----------------------------------------------------------------------------------------

\clearpage

\thispagestyle{empty}

\vspace*{\fill}

{\small
\noindent {\bfseries Spectroscopy of charmed hadrons: facing the latest experimental results with the theory} \\
Tesi di Laurea Magistrale. Università degli Studi di Bari ``Aldo Moro''. \\
Stefano Campanella \\
This document was typeset using \LaTeX{} and other Free Software \\
Version: \today \\
Author's e-mail: \url{stefanocampanella@fastmail.fm}
}

% TABLE OF CONTENTS
\newpage
\tableofcontents* \newpage

\thispagestyle{empty}

\begin{center}
{\LARGE
  \textsc{ \bfseries Spectroscopy of Charmed Hadrons:\\ Facing the Latest Experimental Results \\ with the Theory} \par
}
\HRule
\end{center}
\cleardoublepage

\thispagestyle{empty}
\vspace*{\fill}
\begin{flushright}
  \begin{minipage}[t]{0.6\textwidth}
    \begin{italian}
      Codeste ambiguità, ridondanze e deficienze ricordano quelle che il dottor Franz Kuhn attribuisce a un'enciclopedia cinese che s'intitola \emph{Emporio celeste di conoscimenti benevoli}. Nelle sue remote pagine è scritto che gli animali si dividono in (a) appartenenti all'Imperatore, (b) imbalsamati, (c) ammaestrati, (d) lattonzoli, (e) sirene, (f) favolosi, (g) cani randagi, (h) inclusi in questa classificazione, (i) che s'agitano come pazzi, (j) innumerevoli, (k) disegnati con un pennello finissimo di pelo di cammello, (l) eccetera, (m) che hanno rotto il vaso, (n) che da lontano sembrano mosche. [...] L'impossibilità di penetrare il disegno divino dell'universo non può, tuttavia, dissuaderci dal tracciare disegni umani, anche se li sappiamo provvisor\^{i}.
    \end{italian}\\[\baselineskip]
    (Jorge Luis Borges, \textit{El idioma analítico de John Wilkins})
  \end{minipage}
\end{flushright}
\vspace*{\fill}
\cleardoublepage
