\chapter{Advances in Hadron Spectroscopy}

\epigraph{When the Nobel Prizes were first awarded in 1901, physicists knew of just two objects which are now called ``elementary particles'': the electron and the proton. A deluge of other ``elementary'' particles appeared after 1930; neutron, neutrino, $\mu$ meson (sic), $\pi$ meson, heavier mesons, and various hyperions. I have heard it said that ``the finder of a new elementary particle used to be rewarded by a Nobel Prize, but such a discovery now ought to be punished by a \$ 10'000 fine''}{(Willis Lamb, \emph{Nobel Prize acceptance speech})}

The foundation of our present understanding of the hadronic spectroscopy is the quark model, independently formulated by Gell-Mann and Zweig in 1964. The model described the hadrons known at the time as composite objects of charged fermions, named ``quarks'', having baryon number $B = 1/3$ and electric charge $2/3$ or $-1/3$. The hadrons were organized in multiplets of the $SU(3)$ symmetry group. However, the model implied violations of the Pauli exclusion principle and hence an additional quantum number, later known as \emph{colour}, was attributed to the quarks. Han and Nambu (1965) proposed a model with a supplementary $SU(3)$ group of colour and the idea was promoted to a gauge theory by Bardeen, Fritzsch and Gell-Mann in 1972, with the gluons as gauge bosons of the theory. Therefore, in less than ten years the theory now known as Quantum ChromoDynamics (QCD) took shape (see section \ref{sec:chromodynamics_elements}). Combinations of quarks corresponding to colour singlets could be formed with either $B = 0$ (mesons) or $B = 1$ (baryons). The simplest among such structures are $q \adj{q}$ and $qqq$, $q$ being a quark and $\adj{q}$ an anti-quark. 

Since the inception of the quark model, it was pointed out that more complex structures with the same baryon number of mesons and baryons could exist, such as the ``tetraquark'' mesons $q q \adj{q} \adj{q}$ and the ``pentaquark'' baryons $q q q \adj{q} q$. Moreover, even more structures are possible once taken into account the possibility of valence gluons. A large number of hadrons has been discovered since the formulation of the quark model that fits very well the classification in baryons and mesons (\emph{standard hadrons}). On the other hand, search for the above mentioned different structures (\emph{exotic hadrons}) yielded negative or ambiguous results until recently.

The goal of this chapter is to briefly review some of the advances in exotic as well as ordinary hadron spectroscopy. Particular emphasis will be put on open-charm and open-beauty mesons.

I shall first briefly consider some of the recent progresses in the search for exotica. The status of open-charm and open-beauty mesons will be reported afterwards.

It should be noticed the importance of ordinary hadron spectroscopy in order to exclude the identification of non-standard hadrons with standard ones or identify them as \emph{supernumerary} states, a condition verified when more states with given quantum numbers than expected are experimentally observed.

\section{Exotic Spectroscopy: Observations and Physical Pictures}

Many of the particles discovered in the current era are candidates to be non-standard hadrons \cite{Lebed:2016hpi, Olsen:2017bmm}. Nearly all of these states are charmoniumlike particles, meaning that they contain a $c \adj{c}$ pair and have masses in the same region as conventional charmonium states, while the remaining are bottomoniumlike (defined in a similar way) and pentaquark candidates. 

Such exotic hadron candidates were observed by different experiments devoted to heavy flavour physics and covering a wide range of energies. At the low end, with a c.m.\@ energy between $2$ and $4.6 \ \text{GeV}$ and thus including the production threshold for $c \adj{c}$ pairs, there is the BESIII experiment at the Institute of High Energy Physics in Beijing that operates at the BEPCII $e^+ e^-$ collider. The so-called B-factory experiments Belle and \textsc{BaBar}\footnote{The Belle experiment ran from 1998 to 2010, data analysis is still ongoing. In 2010 the Belle II experiment was approved, which should operate at SuperKEKB accelerator, an upgrade of the KEKB accelerator intended to provide a peak luminosity 40 times larger than KEKB. Similarly, the \textsc{BaBar} experiment run from 1999 to 2008 but the analyses on collected data continue.}, mainly devoted to the production of particles containing $b$ quarks and $b \adj{b}$ quark pairs, operated respectively at the KEKB $e^+ e^-$ collider at KEK and at the PEP-II $e^+ e^-$ collider at SLAC Linear Accelerator Laboratory with c.m.\@ energies at the $\varUpsilon(4S)$ mass, i.e. $10.58 \ \text{GeV}$. The CDF and D0 experiments operated at the $1.96 \ \text{TeV}$ Tevatron $p \adj{p}$ collider. Finally, at the high end, the ATLAS, CMS and LHCb (the first hadron collider experiment dedicated to heavy-flavour physics) operate at the LHC $p p$ collider at CERN, with c.m.\@ energies between $7$ and $13 \ \text{TeV}$. 

The fact that most exotic hadron candidates are quarkoniumlike particles is a posteriori not unexpected. Indeed, the number of different signatures available for such candidates makes them a natural target of this kind of research. A possible signature of exotic structure would be a combination of quantum numbers $J^{P C}$ not allowed by non-relativistic quark models, where $J$ is the spin, $P$ the parity and $C$ the charge conjugation parity. In such models a quarkoniumlike meson with total quark spin $S$ and quark orbital angular momentum $L$ has $J = L + S$, $P = (-1)^{L + 1}$ and $C = (-1)^{L + S}$, and therefore $0^{- -}$ and the series $J^{ (-1)^J (-1)^{J+1}}$ cannot be obtained for any value of $L$ and $S$. Moreover, charged quarkoniumlike states are necessarily exotic, since they must contain more valence quarks than just the neutral pair $c \adj{c}$ or $b \adj{b}$. Finally, the rather good understanding of the quarkonium spectrum (by means of potential models) and the limited availability of unassigned states with masses below $4.5 \ \text{GeV}$, makes possible to pinpoint supernumerary states.

Another reason of the importance of quarkoniumlike particles in the search of unobserved exotic states is the fact that the large mass of the heavy quarks suppress the production of quark anti-quark pairs from vacuum during quark-to-hadron fragmentation processes. Therefore, if a heavy quark and its antiparticle are found in the decay products of an unforeseen meson resonance, then it is likely that they are among the constituents of such a resonance and if they are the only constituents then the resonance is a quarkoniumlike state.

Analogously to the standard hadronic spectrum, in absence of any analytical method for making first-principle calculations of the non-standard one, simplified models have been developed. QCD suggests a colour-motivated classification of non-relativistic models in multi-quark hadrons, such as tetraquark mesons and pentaquark baryons, in hybrid hadrons, which are formed by colourless combinations of quarks and valence gluons, and in glueballs, i.e. mesons composite only of gluons. Other models which do not fit or extend this classification are 
\begin{description}
  \item[QCD diquarks:] it is possible to form coloured combinations of quark pairs that in many respects act as single particles and are called diquarks. These objects are not colour singlets, therefore cannot exist as free particles. However, they could combine with other quarks or diquarks to form colourless particles. This abstraction can be exploited to elaborate models for ordinary as well as exotic hadrons.
  \item[QCD hybrids:] it is known that the colour-force field lines within a meson do not spread out in space, but are tightly confined in ``flux tubes'' that run between the quark and the anti-quark. In excited states it is possible to assign quantum numbers consistent with one or more gluons to the flux tube. In this way some combinations of $J^{P C}$ quantum numbers not allowed to conventional mesons become accessible and such states are called hybrids, meaning that they are formed by a combination of quarks and (valence) gluons.
  \item[Hadronic Molecules:] in this model remnants of the strong force, mediated by Yukawa-type meson exchange, are able to produce bound states between hadrons. Given the analogy with their electromagnetic counterpart, these systems are called hadronic molecules and are expected to have masses near the constituent particles threshold and the quantum numbers of an $s$-wave combination of them. 
  \item[Hadrocharmonium:] in this model, a compact colour-singlet charmonium core state is embedded in a spatially extended ``blob'' or ``cloud'' of light hadronic matter. These two components interact via QCD version of the van der Waals force. It can be shown that the mutual forces in this configuration are strong enough to form bound states if the light hadronic matter is a highly excited resonant state.
  \item[Kinematically Induced Mass Peaks:] in some conditions, kinematic effects can produce resonance-like peaks in the invariant mass distribution of the reaction products and these peaks can be confused with the signal of an unstable hadron.
\end{description}
 
Many of the charmoniumlike states where first observed in $e^+ e^-$ collisions at c.m.\@ energy of the $\varUpsilon(4S)$, hence it is worth reviewing some of the production modes that occur in such processes. 
\begin{description}
  \item[Charmed quark production in $B$-meson decays: ] $\adj{B}$ mesons have a quark content $b \adj{q}$, where $\adj{q}$ is a light anti-quark\footnote{$B_c^\pm$ ($b \adj{c}$ or $\adj{b} c$) mesons have also been observed, however they do not fit in the category of heavy-light mesons this thesis is concerned with.}. When $\adj{q} = \adj{d}, \adj{u}$ their mass is approximately $5.28 \ \text{GeV}$. Therefore, in $e^+ e^-$ annihilations at c.m.\@ energy of $10.58 \ \text{GeV}$, they are produced in $B \adj{B}$ pairs that are nearly at rest and with no accompanying particles. The primary decay channel of such mesons is by weak decay of $b \to c$ though the emission of a virtual $W^-$ boson, which approximately $15\%$ of the times materializes as a $\adj{c}$ and a $s$ quarks. Therefore the $s$ quark can form a $K$ meson with the spectator anti-quark $\adj{q}$ while the $c \adj{c}$ pair can form a charmoniumlike state. This state must have $J^{P C}$ equals to $0^{- +}$, $1^{- -}$ or $1^{+ +}$. 
  \item[Two-photon fusion process:] both the incoming $e^+$ and $e^-$ radiate photons, which then fuse to form a $q \adj{q}$ pair. This is a fourth order QED process and thus its probability is proportional to $e^4_q$, where $e_q$ is the electric charge of $q$. Hence this fact favours the production of $c \adj{c}$ pairs against, say $s$ or $b$ quark anti-quark pairs. It can be shown that the dominant contribution to this process comes from events in which $e^+$ and $e^-$ scatter at small angles and are therefore undetectable. Moreover, for such events the transverse momentum of $c \adj{c}$ is small and these facts all together provide a useful experimental signature. The allowed $J^{P C}$ for $c \adj{c}$ in these processes are $0^{\pm +}$ and $2^{\pm +}$.
  \item[Double charmonium production:]  The production of $J/\psi$ plus a $c \adj{c}$ following $e^+ e^-$ annihilation accounts for approximately $60\%$ of the events with a $J/\psi$ in the final state (prior to the operation of the $B$ factories, this ratio was predicted to be less than $10\%$). This production method favours charmoniumlike systems with $J = 0$.
  \item[Initial State Radiation (ISR):] occasionally one of the two collision partner radiates a high energy photon and subsequently annihilates at reduced c.m.\@ energy. This allows to probe a broad range of reduced c.m.\@ energies and the direct production of charmonium (or charmonium-like) states. Charmoniumlike particles produced in this way have $J^{P C} = 1^{- -}$.
\end{description}

The nomenclature for exotic states is not entirely settled nor consistent. The name currently employed for charmoniumlike states are $X$, $Y$ and $Z$, while bottomoniumlike ones are labelled with a subscript (i.e. $Y_b$ and $Z_b$). Usually, charmoniumlike states first seen in $B$-meson decays are called $X$, those first observed in ISR processes are called $Y$, while charged charmoniumlike states are called $Z$. However, it should be noticed that the Particle Data Group calls all charmoniumlike particles $X$ and bottomoniumlike ones $X_b$. The pentaquark candidates are more consistently indicated in the literature as $P_c$. I now list a few of the observed exotic candidates that are most significant to show the progress in this field.

\paragraph{$X(3872)$ and $Y(4260)$ Neutral Charmoniumlike States} The $X(3872)$ is the first charmoniumlike non-standard candidate that was observed. It was observed in 2003 by the Belle collaboration and appeared as an unforeseen peak in the $\pi^+ \pi^- J/\psi$ invariant mass distribution in $B \to K \pi^+ \pi^- J/\psi$ decays, and later confirmed by many other experiments. Its most peculiar feature is its mass $M_{X(3872)} = 3871.69 \pm 0.17 \ \text{MeV}$ \cite{Patrignani:2016xqp}, which at the current level of precision is indistinguishable from the $D^0 \adj{D}^{* 0}$ threshold. There is strong evidence that this particle has even charge conjugation parity and the LHCb collaboration found that the assignment $J^{P C} = 1^{+ +}$ has the highest likelihood value for $J \leq 4$. Among the states predicted by the quark model, the only one with $J^{P C} = 1^{+ +}$ is the state called $\chi'_{c1}$, the first radial exitation of the $\chi_{c1}$. This assignment is debated, in particular since in this case the observed decay $X(3872) \to \rho J/\psi$ would violate isospin. These facts led to the speculation that this particle could be indeed a non-standard hadron. In particular its mass suggested that the $X(3872)$ could be a hadronic molecule made by a $D^0$ and a $D^{* 0}$. However, also this picture is controversial and the identification of the $X(3872)$ is still uncertain. 

After the discovery of the $X(3872)$ but before its quantum numbers were established, the \textsc{BaBar} collaboration considered the possibility it were a $1^{- -}$ state and searced for its direct production in the ISR process $e^+ e^- \to \gamma_\text{ISR} \pi^+ \pi^- J/\psi$. This analysis led to the conclusion that the quantum numbers of $X(3872)$ are not $J^{P C} = 1^{- -}$. However, during the same analysis it was found an accumulation of events with $\pi^+ \pi^- J/\psi$ invariant masses with a peak near $4.26 \ \text{GeV}$. This resonance was called $Y(4260)$ and its existence was later confirmed by other experiments, CLEO and Belle in the first place. Its production mode ensures that $Y(4260)$ has the same quantum numbers of the photon, i.e. $J^{P C} = 1^{- -}$. However, all the $1^{- -}$ charmonium levels below $4.5 \ \text{GeV}$ are assigned to well-established resonances and hence the $Y(4260)$ qualifies as an exotic candidate. Other properties of the $Y(4260)$ which contrast with the identification with an ordinary charmonium state are the absence of any sign of decays to open-charm mesons and its strong signal in the $\pi^+ \pi^- J/\psi$ decay channel. These facts led to the speculation that the $Y(4260)$ might be a multi-quark meson or a hybrid meson. 
 
\paragraph{$Z(4430)^+$, $Z(4200)^+$ and $\chi_{c1} \pi^+$ Charged Resonances} The $Z(4430)^+$ is the first charged charmonium candidate established and it was observed by Belle in 2007 as a peak in the invariant mass of the $\psi' \pi^+$ system in $\adj{B} \to \psi' \pi^+ K$. At the beginning, \textsc{BaBar} did not find strong evidence of such a resonance but was not able to rule it out either. Subsequent analyses by Belle confirmed the existence of the $Z(4430)^+$, although with larger mass and total width with respect to what suggested by the first analysis. Later its existence was confirmed independently by LHCb and both experiments agree on the $J^P = 1^+$ assignment.

In 2014 the Belle Collaboration found that a satisfactory description of the decays $\adj{B}^0 \to J/\psi \pi^+ K^-$ was possible only including contribution of two resonances in the $J/\psi \pi^+$ channel, one with quantum numbers $J^P = 1^+$ which was called $Z(4200)^+$ and the other corresponding to the $Z(4430)^+$. However, at present, the $Z(4420)^+$ still needs independent confirmation.

In the same analyses the Belle collaboration also reported evidence for two charged $\chi_{c1} \pi^+$ resonances in a Dalitz plot analysis of $\adj{B}^0 \to \chi_{c1} \pi^+ K^-$ decays with axis $M^2(K \pi)$ vs $M^2(\chi_{c1} \pi)$. These resonances were called $Z(4050)^+$ and $Z(4250)^+$ but, as for the $Z(4200)^+$, still await for independent confirmation and a more complete analysis of the six-dimensional phase space (three-body decay).

\paragraph{$Z_b(10610)^+$ and $Z_b(10650)^+$ Charged Bottomoniumlike Resonances} In 2006 the Belle collaboration found an anomalously large $\pi^+ \pi^- \varUpsilon(n_r S)$ ($r = 1,2, 3$) production rate, while searching for a behaviour in the bottomonium system similar to the large $Y(4260) \to \pi^+ \pi^- J/\psi$ signal found by \textsc{BaBar}. This peak in the production rate near c.m.\@ energy of $10.89 \ \text{GeV}$ is very close to the resonance known as $\varUpsilon(10860)$, which is at present identified with $\varUpsilon(5 S)$. In 2012 further analysis of the same collaboration in the same region led to the discovery of two previously unseen bottomonium states, the $h_b(1P)$ and $h_b(2P)$, and of two charged bottomoniumlike states $Z_b(10610)^+$ and $Z_b(10650)^+$, which appeared in intermediate processes $e^+ e^- \to \pi^\pm Z_b(10610)^\mp$ and $e^+ e^- \to \pi^\pm Z_b(10650)^\mp$. Later studies assigned $J^P = 1^+$ to both $Z_b(10610)^+$ and $Z_b(10650)^+$ and observed a neutral partner of the $Z_b(10610)^+$, where the ratio of cross sections of the charged and associated neutral processes are consistent with expectations for an isovector $Z_b(10610)^+$. The $Z_b^+$ masses are just above the production threshold for $B \adj{B}^*$ and $B^* \adj{B}^*$ respectively. This fact, together with their quantum numbers, motivated the experimental research of decays in such channels and theoretical speculations about their interpretation as $s$-wave hadronic molecules. Indeed, in 2016 the Belle collaboration observed the decays $Z_b(10610) \to B \adj{B}^*$ and $Z_b(10650) \to B^* \adj{B}^*$.

\paragraph{$P_c(4380)^+$ and $P_c(4450)^+$ Pentaquark Candidates} In 2015 the LHCb collaboration reported evidence for pentaquarklike structures with a minimal quark content of $uudc\adj{c}$ in the context of a study of the process $\Lambda^0_b \to J/\psi p K^-$ resonances. In addition to contributes from many established $\Lambda^* \to K^- p$ intermediate resonances, the data contained a narrow peak in the $J/\psi p$ invariant mass. It was performed a kinematically complete six-dimensional amplitude fit finding that it was necessary to add two non-standard $P_c^+ \to J/\psi p$ pentaquark contributions to reproduce this structure in the $M(J/\psi p)$ invariant mass. These two states were called $P_c(4380)^+$ and $P_c(4450)^+$. Although the quantum numbers of these states were not uniquely determined, LHCb was able to restrict their value to the combinations $J_{P_c^+(4380)} = 3/2$ and $J_{P_c^+(4450)} = 5/2$ or $J_{P_c^+(4380)} = 5/2$ and $J_{P_c^+(4450)} = 3/2$, with $P_c^+(4380)$ and $P_c^+(4380)$ having of opposite parity in the two cases.

\section{The Present State of Heavy Meson Spectroscopy}

The expression charmed hadrons, or open-charm hadrons, means hadrons containing a single charm quark $c$. Hadrons containing more than one $c$ quark are called doubly charmed, while hadrons in which $c$ appears in quark anti-quark pairs $c \adj{c}$ are said to have hidden charm. Hadrons in which the other constituents are light quarks (i.e. $u$, $d$ or $s$) are of particular interest. Indeed, given the great difference between the masses of heavy and light quarks, several approximations can be made in the description of such hadrons that can be consistently formulated as effective theories. Among the baryons there are many different known charmed states, which are usually labelled with a subscript $c$ (e.g. $\Lambda_c$, $\Sigma_c$, etc.), while among the mesons one can distinguish the $D$ and $D_s$ mesons, which respectively have strangeness $S = 0$ or $S = 1$. Analogous considerations hold for beauty hadrons, in which case the mesons with open beauty are the $B$ and $B_s$ mesons. Together the $D_{(s)}$ and $B_{(s)}$ mesons are referred to as heavy mesons, or sometimes as heavy-light mesons, when one wants to stress the fact that the other quark is a light one. The following of this section will be concerned with an overview of the present state of heavy meson spectroscopy, with a focus on charmed mesons, the subject of this thesis.

\begin{table}
  \centering
  \begin{tabular}{c c c c}
    \toprule
    $c \adj{q}$ & mass (MeV) & $\tau$ (s) & $J^P$ \\
    \midrule
    $D^0$ & $1864.83 \pm 0.05 $ & $\left( 4.101 \pm 0.08 \right) \times 10^{-13}$ & $0^-$ \\
    $D^+$ & $1869.59 \pm 0.09$ & $\left( 1.040 \pm 0.007 \right) \times 10^{-12}$ & $0^-$ \\ 
    \midrule[\heavyrulewidth]
    $c \adj{q}$ & mass (MeV) & $\Gamma$ (MeV) & $J^P$ \\ 
    \midrule
    $D^{* 0}$ & $2006.85 \pm 0.05$ & $< 2.1$ & $1^-$ \\
    $D^{* +}$ & $2010.26 \pm 0.05$ & $0.083 \pm 0.002$ & $1^-$ \\
    \addlinespace
    $D_0^*(2400)^0$ & $2318 \pm 29$ & $267 \pm 40$ & $0^+$ \\
    $D_0^*(2400)^+$ & $2351 \pm 7$ & $230 \pm 17$ & $0^+$ \\
    \addlinespace
    $D'_{1}(2430)^0$ & $2427 \pm 40$ & $384^{+130}_{-110}$ & $1^+$ \\
    \addlinespace
    $D_1(2420)^0$ & $2420.8 \pm 0.5$ & $31.7 \pm 2.5$ & $1^+$ \\
    $D_1(2420)^+$ & $2423.2 \pm 2.4$ &  $25 \pm 6$ & $1^+$ \\
    \addlinespace
    $D^*_2(2460)^0$ & $2460.7 \pm 0.4$ & $ 47.5 \pm 1.1$ & $2^+$ \\
    $D^*_2(2460)^+$ & $2465.4 \pm 1.3$ & $46.7 \pm 1.2$ & $2^+$ \\
    \bottomrule
  \end{tabular}
  \caption{Masses, widths and spin-parity of the established open-charm mesons \cite{Patrignani:2016xqp}.}
  \label{tab:D_established}
\end{table}

\subsection{Open-charm Spectroscopy}

Well established charmed mesons are the pseudoscalar $D$ meson, the vector $D^*$, the axial vector $D_1(2420)$ and the $J^P = 2^+$ $D^*_2(2460)$. However, today the spectrum of observed states is far richer. The modern history of charmed meson spectroscopy can be made to begin in the 2000, with the evidence gained by the CLEO collaboration of two new broad states \cite{Anderson:1999wn}. 

In 2003 the Belle collaboration confirmed the latter, with the observation of two broad states near $2.4 \ \text{GeV}$ obtained through the Dalitz plot analysis of charged $B$ decays $B^\pm \to D^{(*) \mp} \pi^\pm \pi^\pm$ \cite{Abe:2003zm}, where the selection of final states with pions with the same sign prevented them from forming bound states (thus making the analysis simpler). Indeed, contributions from the $D_2^{* 0}$ and a broad scalar resonance, the $D^*_0(2400)^0$, were needed in order to describe the data in the channel $D \pi \pi$, while contribution from the $D_2^{* 0}$, the $D^0_1$ and a broad axial excited $D$ meson, the $D'_1(2430)^0$, were needed in the channel $D^* \pi \pi$. It was also proposed that there could be a mixing of the two $1^+$ states, i.e. $D_1$ and $D'_1$, and indeed Belle measured a small mixing angle $\omega = 0.10 \pm 0.04 \ \text{rad} \approx 6 \ \text{deg}$, suggesting that such mixing can be safely neglected. Particles with quantum numbers and broad widths\footnote{Associated to $s$-wave decays, in contrast with the narrow $D_2^*$ and $D_1$, which decay through $d$-wave.} of these new states were expected in the quark model, but in a different mass region. 

Shortly after, in the 2004, another confirmation of the existence of the $D^*_0(2400)$ (both in the neutral and charged version) came from the analysis of data collected by the FOCUS experiment\footnote{FOCUS, aka E831, was an upgraded version of the experiment E687, where a forward multi-particle spectrometer was used to to investigate the interactions of high energy photons on a segmented BeO target. The photon beam was derived from the bremmstrahlung of secondary electrons ($< 250 \ \text{GeV}$) produced from the Tevatron proton beam.} at Fermilab in the 1996-1997 fixed-target run. Finally, the $D^*_0(2400)^0$ was observed also by \textsc{BaBar} \cite{Aubert:2009wg} in 2009 and the $D^*_0(2400)^-$ by LHCb in 2015 \cite{Aaij:2015kqa,Aaij:2015sqa} while the $D'_1(2430)$ by \textsc{BaBar} \cite{Aubert:2006zb} in 2006. In light of these observations, the existence of these states can be safely considered as established (thus are included in table \ref{tab:D_established}).

\begin{table}
  \centering
  \begin{tabular}{c c c c}
    \toprule
    Resonance & mass (MeV) & $\Gamma$ (MeV) & $J^P$ \\ 
    \midrule
    $D(2550)^0 $ & $2539.4 \pm 4.5 \pm 6.8$ & $130 \pm 12 \pm 13$ & $0^-$ \\
    \addlinespace
    $D^*(2600)^0$ & $2608.7 \pm 2.4 \pm 2.5$ & $93 \pm 6 \pm 13$ & natural \\
    $D^*(2600)^\pm$ & $2621.3 \pm 3.7 \pm 4.2$ & $93 \ \text{(fixed)}$ & natural \\
    \addlinespace
    $D(2750)^0$ & $2752.4 \pm 1.7 \pm 2.7$ & $71 \pm 6 \pm 11$ & \\
    \addlinespace
    $D^*(2760)^0$ & $2763.3 \pm 2.3 \pm 2.3$ & $60.9  \pm 5.1 \pm 3.6$ & natural \\
    $D^*(2760)^\pm$ & $2769.7 \pm 3.8 \pm 1.5$ & $60.9 \ \text{(fixed)}$ & natural \\
    \bottomrule
  \end{tabular}
  \caption{Mass, width and spin-parity of new open-charm meson candidates observed by \textsc{BaBar} in 2010 \cite{delAmoSanchez:2010vq}.}
  \label{tab:BaBar_2010}
\end{table}

In 2010 the \textsc{BaBar} collaboration made an inclusive production study \cite{delAmoSanchez:2010vq} of processes of the type $e^+ e^- \to c \adj{c} \to D^{(*)} \pi X$ at SLAC-PEPII, where $X$ is any additional system. The size of the event sample was approximately $6 \times 10^8$ events at c.m.\@ energy of $10.58 \ \text{Gev}$, corresponding to $\approx 450 \text{fb}^{-1}$. The $D \pi$ system was investigated in the neutral $D^+ \pi^-$ and charged $D^0 \pi^+$ channels, were the $D^+$ was reconstructed through the decay $D^+ \to K^- \pi^+ \pi^+$ and the $D^0$ through $D^0 \to K^- \pi^+$. The $D^* \pi$ system was investigated only in the neutral channel $D^{* +} \pi^-$ and reconstructed through the decay $D^{* +} \to D^0 \pi^+_\text{s}$, were $\pi_\text{s}$ is a slow pion constrained to having a production vertex in proximity of the $D^0$, which was reconstructed through $D^0 \to K^- \pi^+$ and $D^0 \to K^- \pi^+ \pi^- \pi^+$ decay modes.

The $D \pi$ system was studied looking for structures for the mass variables $M \left( D^+ \pi^- \right) = m \left( K^- \pi^+ \pi^+ \pi^- \right) - m \left( K^- \pi^+ \pi^+ \right) + m_{D^+}$ and $M \left( D^0 \pi^+ \right) = m \left( K^- \pi^+ \pi^+ \right) - m \left( K^- \pi^+ \right) + m_{D^0}$. In this amplitude analysis, the peaks for the $D^*_2(2460)$ isospin triplet, and two unforeseen structures around $2.6 \ \text{GeV}$ and $2.75 \ \text{GeV}$, the $D^*(2600)$ and the $D^*(2760)$, were observed. It should be noticed that spin-parity conservation implies that states strongly decaying to $D \pi$ must have natural parity, moreover, if they also decay to $D^* \pi$ then the $J^P = 0^+$ assignment is ruled out. Also, because the observed $D^0 \pi^+$ mass spectrum was similar to the one of $D^+ \pi^-$, but the background was larger and the statistical precision lower, assuming isospin symmetry, the width of the new resonances were fixed to the values measured in the $D^+ \pi^-$ channel (see table \ref{tab:BaBar_2010}).

The $D^* \pi$ system was analysed looking for structures for the mass variable $M \left( D^{*+} \pi^- \right) = m \left( K^- \pi^+ \pi^+ \pi^- \pi^+_\text{s} \pi^- \right) - \left( K^- \pi^+ \pi^+ \pi^- \pi^+_\text{s} \right) + m_{D^{* +}}$, which brought to the observation of the peaks corresponding $D_1(2420)^0$ and $D^*_2(2460)^0$; a peak near $2.6 \ \text{GeV}$, identified with the $D^*(2600)$ found in the $D \pi$ system, and another peak near $2.75 \ \text{GeV}$, called the $D(2750)^0$.

In order to fix the spin-parity of the $D^*(2600)^0$ and $D^*(2750)^0$ a study of the helicity angle $\theta_h$ was conducted. The latter is defined as the angle between the flight of the fast primary pion in the c.m.\@ frame of $D^* \pi$ and the flight of the slow secondary pion, produced in the decay of the $D^*$, in the frame in which $D^*$ is at rest. Signals from natural parity states are proportional to $\sin^2(\theta_h)$ (except for $J^P = 0^+$, which is ruled out), the ones from $0^-$ states are proportional to $\cos^2(\theta_h)$ and signals from all other unnatural parity states are proportional to $1 - h \cos^2(\theta_h)$, where $h > 0$ is a free parameter.

The analysis was initially conducted assuming only two signals at $2.6 \ \text{GeV}$ and $2.75 \ \text{GeV}$. However, these hypothesis failed to describe well the data and this led to the inclusion of another state in the mass region of the $D^*(2660)$, the $D(2550)^0$. This analysis yielded results compatible with natural parity assignment to the $D^*(2600)^0$, which supported the identification of the latter as the same state found in the $D \pi$ channel. The $D(2550)^0$ was found to be consistent with $J^P = 0^-$, while the assignment of natural or unnatural parity to the $D(2750)^0$ could not be decided.

Finally, \textsc{BaBar} measured the following relative branching ratios
\begin{subequations}
  \begin{align}
    \frac{\symcal{B} \left( D^*(2600)^0 \to D^+ \pi^- \right)}{\symcal{B} \left(  D^*(2600)^0 \to D^{* +} \pi^- \right)} &= 0.32 \pm 0.02 \pm 0.09 \ , \\
    \frac{\symcal{B} \left( D^*(2760)^0 \to D^+ \pi^- \right)}{\symcal{B} \left(  D^*(2750)^0 \to D^{* +} \pi^- \right)} &= 0.42 \pm 0.05 \pm 0.11 \ .
  \end{align}
\end{subequations}

The assignment $J^P = 3^-$ for the $D^*_J(2760)$ was subsequently fixed by a Dalitz plot analysis of the LHCb collaboration \cite{Aaij:2015sqa}.

\begin{table}
  \centering
  \begin{tabular}{c c c c}
    \toprule
    Resonance & mass (MeV) & $\Gamma$ (MeV) & $J^P$ \\ 
    \midrule
    $D_J(2580)^0$     & $2579.5 \pm 3.4 \pm 5.5$  & $177.5 \pm 17.7 \pm 46.0$ & unnatural \\
    \addlinespace
    $D^*_J(2650)^0$   & $2649.2 \pm 3.5 \pm 3.5$  & $140.2 \pm 17.1 \pm 18.6$ & natural \\
    \addlinespace
    $D_J(2740)^0$     & $2737.0 \pm 3.5 \pm 11.2$ & $73.2 \pm 13.4 \pm 25.0$  & unnatural \\
    \addlinespace
    $D^*_J(2760)^0$   & $2761.1 \pm 5.1 \pm 6.5$  & $74.4 \pm 4.3 \pm 37.0$   & natural \\
    $D^*_J(2760)^0$   & $2760.1 \pm 1.1 \pm 3.7$  & $74.4 \pm 3.4 \pm 19.1$   & natural \\
    $D^*_J(2760)^\pm$ & $2771.7 \pm 1.7 \pm 3.8$  & $66.7 \pm 6.6 \pm 10.5$   & natural \\
    \addlinespace
    $D_J(3000)^0$     & $2971.8 \pm 8.7$          & $188.1 \pm 44.8$          & unnatural \\
    \addlinespace
    $D^*_J(3000)^0$   & $3008.1 \pm 4.0$          & $110.5 \pm 11.5$          & natural \\
    $D^*_J(3000)^\pm$ & $3008.1 \ \text{(fixed)}$ & $110.5 \ \text{(fixed)}$  & natural \\
    \bottomrule
  \end{tabular}
  \caption{Masses, widths and spin-parity of new open-charm meson candidates observed by LHCb in 2013 \cite{Aaij:2013sza}.}
  \label{tab:LHCb_2013}
\end{table}

In 2013 the LHCb collaboration performed a search for excited $D$ mesons in $p p$ collisions at c.m.\@ energy of $7 \ \text{TeV}$. The data sample consisted of $1 \ \text{fb}^{-1}$ of events corresponding to inclusive reactions of the type $p p \to D^+ \pi^- X$, $p p \to D^0 \pi^+ X$ and $p p \to D^{* +} \pi^- X$. These processes were investigated through the same decay channels and mass variables already outlined for the \textsc{BaBar} study of 2010.

In order to analyse the $D^{* +} \pi^-$ mass spectrum three sub-samples were extracted: a \emph{natural parity sample} with $\cos{(\theta_h)}< 0.5$ ($68.8\%$ of the initial sample), an \emph{unnatural parity sample} with $\cos{(\theta_h)}> 0.5$ ($31.2\%$) and an \emph{enhanced unnatural parity sample} with $\cos{(\theta_h)}> 0.75$ ($8.6 \%$). In the enhanced unnatural sample a resonance consistent with the $D_1(2420)^0$ was observed as well as other three resonances: the $D_J(2580)^0$, the $D_J(2740)^0$ and the $D_J(3000)$, hence all with unnatural parity. The masses and widths of these resonances are fixed in the fit for the natural parity sample, which exhibits the prominent contribution of the $D^*_2(2460)^0$ and the minor ones of the $D_J(2580)^0$, $D_J(2740)^0$ and $D_J(3000)^0$. Within this sample two new resonances were observed, the $D^*_J(2650)^0$ and the $D^*_J(2760)^0$, which thus have natural parity. The unnatural parity sample and the total sample were used as a cross-check, with parameters fixed by the previous fits. Finally, the spin-parity assignments were checked against the analysis of the helicity angle (similar to those performed in the \textsc{BaBar} study) confirming previous result (the $J^P = 0^-$ hypothesis was also considered for the $D_J(2580)^0$).

In the charged and neutral $D \pi$ systems, the charged partner of the $D^*_J(2760)^0$ was observed together with two other nonestablished states, the $D^*_J(3000)^0$ and the $D^*_J(3000)^\pm$. The mass and width of the latter were fixed to the value of the former, due to large background and poor statistics, assuming isospin symmetry. However, the LHCb collaborations warns that the properties of these last states are uncertain and that they may correspond to superposition of several states in that mass region.

\begin{table}
  \centering
  \begin{tabular}{c c c c}
    \toprule
    Resonance & mass (MeV) & $\Gamma$ (MeV) & $J^P$ \\ 
    \midrule
    %$D^*_2(2460)^0$   & $2463.7 \pm 0.4 \pm 0.4 \pm 0.6$  & $47.0  \pm 0.8 \pm 0.9 \pm 0.3$  & $2^+$ \\
    $D^*_1(2680)^0$   & $2681.1 \pm 5.6 \pm 4.9 \pm 13.1$ & $186.7 \pm 8.5 \pm 8.6 \pm 8.2$  & $1^-$ \\
    $D^*_3(2760)^0$   & $2775.5 \pm 4.5 \pm 4.5 \pm 4.7$  & $95.3  \pm 9.6 \pm 7.9 \pm 33.1$ & $3^-$ \\
    $D^*_2(3000)^0$   & $3214   \pm 29  \pm 33  \pm 36$   & $186   \pm 39  \pm 34  \pm 63$   & $2^+$ \\
    \bottomrule
  \end{tabular}
  \caption{Masses, widths and spin-parity of new open-charm meson candidates observed by LHCb in 2016 \cite{Aaij:2016fma}.}
  \label{tab:LHCb_2016}
\end{table}

In 2016 the LHCb collaboration made a Dalitz plot analysis of $B^- \to D^+ \pi^- \pi^-$ decays \cite{Aaij:2016fma} looking for structure in the neutral $D^+ \pi^-$ channel, where the $D^+$ was reconstructed through the decay $D^+ \to K^- \pi^+ \pi^+$, using a data sample correspondint to $3 \ \text{fb}^{-1}$ of $p p$ collisions at c.m.\@ energy of $7 \ \text{TeV}$ (2011) and $8 \ \text{TeV}$ (2012). Apart from the $D^*_2(2460)^0$, three other resonances with masses near $2.68, \ 2.76 \text{ and } 3.0 \ \text{GeV}$ were observed. The analyses allowed to determine the spin of these resonances and, because they must belong to the natural parity series, to fix the parity. These resonances are the $D^*_1(2680)$, the $D^*_3(2760)^0$ and the $D^*_2(3000)$ and their properties are reported in table \ref{tab:LHCb_2016}. It is not clear if the $D^*_1(2680)$ or the $D^*_2(3000)$ can be identified with previously observed states, while the $D^*_3(2760)$ seems to be consistent with previous measurement \cite{delAmoSanchez:2010vq,Aaij:2013sza,Aaij:2016fma}.

\begin{table}
  \centering
  \begin{tabular}{c c c c}
    \toprule
    $c \adj{s}$ & mass (MeV) & $\tau$ (s) & $J^P$ \\ 
    \midrule
    $D^+_s$ & $1968.28 \pm 0.10$ & $\left( 5.00 \pm 0.07 \right) \times 10^{-13}$ & $0^-$ \\
    \midrule[\heavyrulewidth]
    $c \adj{s}$ & mass (MeV) & $\Gamma$ (MeV) & $J^P$ \\ 
    \midrule
    $D^{* +}_s$ & $2112.1 \pm 0.4$ & $< 1.9$ & $1^-$ \\
    $D^*_{s0}(2317)^+$ & $2317.7 \pm 0.6$ & $<3.8$ & $0^+$ \\
    $D'_{s1}(2460)^+$ & $2459.5 \pm 0.6$ & $<3.5$ & $1^+$ \\
    $D_{s1}(2536)^+$ & $2535.10 \pm 0.06$ & $0.92 \pm 0.05$ & $1^+$ \\
    $D^*_{s2}(2573)^+$ & $2569.1 \pm 0.8$ & $16.9 \pm 0.8$ & $2^+$ \\
    $D^*_{s1}(2700)^+$ & $2708.3^{+4.0}_{-3.4}$ & $120 \pm 11$ & $1^-$ \\
    $D^*_{s1}(2860)^+$ & $2859 \pm 27$ & $159 \pm 80$ & $1^-$ \\
    $D^*_{s3}(2860)^+$ & $2860 \pm 7$ & $53 \pm 10$ & $3^-$ \\
    $D^*_{sJ}(3040)^+$ & $3044^{+31}_{-9}$ & $239 \pm 60$ & \\
    \bottomrule
  \end{tabular}
  \caption{Masses, widths and spin-parity of observed $c \adj{s}$ mesons \cite{Patrignani:2016xqp}.}
  \label{tab:Ds_mesons}
\end{table}

The spectrum of observed strange open-charm mesons has seen as well many improvements in recent years. Before the operation of the B-factories only four charmed mesons with strangeness were known: the pseudoscalar $D_s$ meson, the vector $D^*_s$, the axial $D_{s 1}(2536)$ and the $J^P = 2^+$ $D^*_{s 2}(2573)$. The $D^*_s$ has a narrow width and decays mainly through $D_s \gamma$.The $D_{s 1}(2536)$ and the $D_{s 2}(2573)$ decay respectively in the $D^* K$ and $D K$ channels.

In 2003 the \textsc{BaBar} collaboration observed a very narrow state with mass near $2.32 \ \text{GeV}$ \cite{Aubert:2003fg} in $e^+ e^-$ collisions at c.m.\@ energy of the $\varUpsilon(4 s)$ ($\approx 10.6 \ \text{GeV}$), which was called the $D^*_{sJ}(2317)$. The data sample consisted of $91 \ \text{fb}^{-1}$ of events related to inclusive processes $e^+ e^- \to D_s \pi^0 X$, where $X$ is an arbitrary collection of particles. The $D_s^+$ was reconstructed through the decay $D_s^+ \to K^+ K^- \pi^+$ while the $\pi^0$ through $\pi \to \gamma \gamma$. The decay mode indicated natural parity and hence its low mass suggested the assignment $0^+$ (i.e. the lowest available state). The narrow width was consistent with decays violating isospin conservation. However, if tentative $J^P = 0^+$ assignment is confirmed, the low mass, small width, and decay mode of $D^*_{sJ} (2317)$ are quite different from those predicted by potential models.

Shortly after, the CLEO experiment confirmed the existence of this state and reported of another one with mass near $2.46 \ \text{GeV}$ \cite{Besson:2003cp}, which was called the $D_{s J}(2463)$ (a.k.a.\@ $D_{s J}(2460)$). The search was conducted on data collected by the CLEOII detector and relative to the same inclusive processes and c.m.\@ energy of the \textsc{BaBar} study. Indeed, this search was motivated by the latter. The reconstruction of the $D_s^+$ in these case was through the chain of decays $D_s^+ \to \phi \pi^+$ and $\phi \to K^+ K^-$. In the same year, the Belle collaboration observed these states in inclusive processes \cite{Abe:2003jk} and in $B$ decays \cite{Krokovny:2003zq} ($B \to \adj{D} D_{s J} (2317)$ and $B \to \adj{D} D_{s J}(2457)$, in the article indicated as $D_{s J}(2457)$).

The spin-parity of the $D^*_{s J}(2317)$ and of the $D_{s J}(2460)$, or $D^*_{s 0}(2320)$ and $D^*_{s 1}(2460)$ as the PDG calls these states, is not firmly established. For the $D^*_{s 0}$, experimental data have not ruled out higher spin from the natural series. There is also evidence for assignment of $J^P = 1^+$ to the $D_{s1}(2460)$, which nonetheless cannot be considered definitive \cite{Aubert:2006bk}. However, their width and mass splitting is consistent with what is expected from their supposed identification. 

It is remarkable that their masses are much smaller than suggested by the quark model, as can be easily seen comparing other corresponding pairs of heavy mesons with and without strangeness. Indeed, they are almost degenerate with their non-strange partners whereas others differ by about $100 \ \text{MeV}$, which accounts for the mass difference between the strange and the down or up quarks.

In 2006 the \textsc{BaBar} collaboration observed a new resonance in $D^0 K^+$ and $D^+ K^0_S$ systems \cite{Aubert:2006mh} during amplitude analysis of $e^+ e^-$ collisions corresponding to $240 \ \text{fb}^{-1}$ at the c.m.\@ energy of $10.6 \ \text{GeV}$, which was called $D^*_{s J}(2860)^+$. Moreover, in the same analysis it was observed a signal which could not be described by enhancements produced by two or three body decays of known narrow resonances where one of the decay products was missed (so-called \emph{reflections}), which was indicated as $X(2690)^+$ and had a rather large width ($\approx 100 \ \text{MeV}$). The latter was later identified with a vector $D_s$ meson excitation, called $D^*_{s 1}(2700)$, observed by Belle in 2008 \cite{Brodzicka:2007aa} during the analysis of approximately $5 \times 10^8$ $B \adj{B}$ events at $10.6 \ \text{GeV}$, through $B^+ \to \adj{D}^0 D^{**}_s \to \adj{D}^0 D^0 K^+$ decays (where $D_s^{**}$ is a generic state).

The existence of these two new $D_s$ mesons was confirmed by \textsc{BaBar} in 2009 \cite{Aubert:2009ah} in a study of the decays $D^*_{s 1}(2710)^+ \to D^* K$ and $D^*_{s J}(2860)^+ \to D^* K$. An helicity angle analysis of these decays allowed to confirm the natural spin-parity of the $D^*_{s 1}(2700)$ and of the $D^*_{s J}(2860)$. Moreover, it was possible to calculated the relative branching ratios
\begin{subequations}
  \begin{align}
    \frac{\symcal{B} \left( D^*_{s 1}(2700)^+ \to D^* K \right)}{\symcal{B} \left(  D^*_{s 1}(2700)^+ \to D K \right)} &= 0.91 \pm 0.13 \pm 0.12 \ , \\
    \frac{\symcal{B} \left( D^*_{s J}(2860)^+ \to D^* K \right)}{\symcal{B} \left(  D^*_{s J}(2850)^+ \to D K \right)} &= 1.10 \pm 0.15 \pm 0.19 \ ,
  \end{align}
\end{subequations}
where $\frac{\symcal{B} \left( D^{* * +}_s \to D^* K \right)}{\symcal{B} \left( D^{* * +}_s \to D K \right)}$ is the weighted average of $\frac{\symcal{B} \left( D^{* * +}_s \to D^{* 0} K^+ \right)}{\symcal{B} \left( D^{* * +}_s \to D^0 K^+ \right)}$ and $\frac{\symcal{B} \left( D^{* * +}_s \to D^{* +} K^0_S \right)}{\symcal{B} \left( D^{* * +}_s \to D^+ K^0_S \right)}$. Finally, in this study it was observed another a broad state near $3 \ \text{GeV}$, the $D_{s J}(3040)$.

The existence of the $D^*_{s 1}(2700)^+$ and of the $D_{s J}(2860)^+$ was confirmed by LHCb in 2012 \cite{Aaij:2012pc} in inclusive amplitude analysis of $D^+ K^0_S$ and $D^0 K^+$ systems in $p p$ collisions relative to a data sample of $1 \ \text{fb}^{-1}$ at $7 \ \text{TeV}$. However, they did not observed any statistically significant $D_{s J}$ resonance in the mass region above $3 \ \text{GeV}$. In addition, a Dalitz plot analysis of $B^0 \to D^- D^0 K^+$ and $B^+ \to \adj{D}^0 D^0 K^+$ decays conducted by \textsc{BaBar} in 2014 \cite{Lees:2014abp} confirmed the assignment $J^P = 1^-$ to the $D^*_{s 1}(2700)^+$ (but did not observed the $D^*_{s J}(2860)$ nor the $D_{s J}(3040)$).

The spin-parity assignment of the $D^*_{s J}(2860)^+$ remained an open question until the 2014, when the LHCb collaboration performed a Dalitz plot analysis of the $B^0_s \to \adj{D}^0 K^- \pi^+$ decay \cite{Aaij:2014xza} using a sample corresponding to $3.0 \ \text{fb}^{-1}$ of $p p$  collision data at c.m.\@ energy of $7 \ \text{TeV}$. The signal in the $m(\adj{D}^0 K^-) \approx 2.86 \ \text{GeV}$ was found to be an admixture of spin 1 and spin 3 resonances. Therefore the $D^*_{s J} (2860)$ state previously seen consisted of at least two particles, the $D^*_{s 1}(2860)$ and the $D^*_{s 3}(2860)$. The latter is the first state among heavy mesons to which experimental evidence assigned spin 3.

\subsection{Open-beauty spectroscopy}

\begin{table}
  \centering
  \subfloat[]{
    %\footnotesize
    \begin{tabular}{c c c c c c c}
      \toprule
      $b \adj{q}$ & mass (MeV) & $\tau$ (s) & $J^P$ \\
      \midrule
      $B^0$ & $5279.63 \pm 0.15$ & $\left( 153.0 \pm 0.4 \right) \times 10^{-14}$ & $0^-$ \\
      $B^-$ & $5279.32 \pm 0.14$ & $\left( 163.8 \pm 0.4 \right) \times 10^{-14}$ & $0^-$ \\
      \midrule[\heavyrulewidth]
      $b \adj{q}$ & mass (MeV) & $\Gamma$ (MeV) & $J^P$ \\
      \midrule
      $B^*$ & $5324.65 \pm 0.25$ & & $1^-$ \\
      \addlinespace
      $B_1(5721)^0$ & $5726.0 \pm 1.3$ & $27.5 \pm 3.4$ & $1^+$ \\
      $B_1(5721)^-$ & $5725.9^{+2.5}_{-2.7}$ & $31 \pm 6$ & $1^+$ \\
      \addlinespace
      $B^*_2(5747)^0$ & $5739.5 \pm 0.7$ & $24.2 \pm 1.7$ & $2^+$ \\
      $B^*_2(5747)^-$ & $5737.2 \pm 0.7$ & $20 \pm 5$ & $2^+$ \\
      \bottomrule
    \end{tabular}
    \label{tab:B}
  }
  \quad
  \subfloat[]{
    %\footnotesize
    \begin{tabular}{c c c c c c c}
      \toprule
      $b \adj{s}$ & mass (MeV) & $\tau$ (s) & $J^P$ \\
      \midrule
      $B^0_s$ & $5366.89 \pm 0.19$ & $\left( 1.505 \pm 0.005 \right) \times 10^{-12}$ & $0^-$ \\
      \midrule[\heavyrulewidth]
      $b \adj{s}$ & mass (MeV) & $\Gamma$ (MeV) & $J^P$ \\
      \midrule
      $B^{* 0}_s$ & $5415.4^{+ 1.8}_{-1.5}$ & & $1^-$ \\
      $B_{s1}(5830)^0$ & $5828.63 \pm 0.27$ & $0.5 \pm 0.4$ & $1^+$ \\
      $B^*_{s2}(5840)^0$ & $5839.85 \pm 0.17$ & $1.47 \pm 0.33$ & $2^+$ \\
      \bottomrule
    \end{tabular}
    \label{tab:Bs}
  }
  \caption{Measured masses and widths of the observed open-beauty mesons \cite{Patrignani:2016xqp}}%, quantum numbers are quark model predictions.}
  \label{tab:B_and_Bs}
\end{table}

The spectrum of open-beauty mesons is less rich than the charmed one. The fundamental state $B$ as well as the vector $B^*$ are well established. The first evidences of open-beauty resonances were gained by several collaborations at LEP (OPAL, DELPHI, ALEPH) in the second half of the 90's. However, in these analyses of inclusive or semi-exclusive $B$ decays, the separation of states was impossible. In 2007 the D0 collaboration observed two $J^P = 1^+ \text{ and } 2^+$ states, the $B_1$ and the $B^*_2$, later confirmed by CDF. Observations in the strange and non-strange sector are reported in table \ref{tab:B_and_Bs}. Moreover, all these states were confirmed by LHCb.

\section{Afterword and a Note on the Baryon Sector}

The present wealth of observed heavy mesons motivates phenomenological studies to classify them. This kind of studies corroborates and supports first principle calculations. Moreover the experimental analyses are not always able to bring a complete description of observed states, like in prompt production studies, that can only determine if the spin-parity of a given meson belongs to the natural or unnatural series, but cannot fix the precise value of $J^P$ quantum numbers. Furthermore in such cases, the phenomenological approach can predict or give criteria for assignment of those quantum numbers. However, these studies not only can guide in the identification of observed states, but also can suggest where to look for yet unobserved ones.

Even though this thesis is only concerned with the meson case and thus I will not review the general state of the baryon spectroscopy, I would like to conclude this overview by mentioning the latest important results in the baryon sector. Indeed, although the heavy mesons spectroscopy counts a larger number of observed states, the spectroscopy of baryons containing a heavy quark has seen great improvements in recent years and, notwithstanding its intrinsic experimental challenges, it is a promising field of experimental research due to the work at the LHCb and Belle II facilities. 

The spectrum of $\Lambda_c^+$, $\Sigma_c$ and $\Xi_c$ particles, in spite of considerable experimental progress, is largely unknown. Moreover, until the 2017, the only $\Omega_c$ baryon states (with quark content $css$ and isospin zero) known were the $\Omega_c^0$ and the $\Omega_c^0(2770)$, presumed to be the $J^P = \left. 1/2 \right.^+ \text{ and } \left. 3/2 \right.^+$ states. In 2017, the $\Xi_c^+ K^-$ mass spectrum was investigated by the LHCb experiment \cite{Aaij:2017nav}. A large high-purity sample of $\Xi_c$ baryons was reconstructed in the Cabibbo-suppressed decay mode $p K^- \pi^+$ and five new, narrow excited states were observed: the $\Omega_c(3000)^0$, the $\Omega_c(3050)^0$, the $\Omega_c(3066)^0$, the $\Omega^c(3090)^0$ and the $\Omega_c(3119)^0$. In the same year a highly significant structure was observed in the $\Lambda_c^+ K^- \pi^+ \pi^+$ mass spectrum \cite{Aaij:2017ueg}, where the $\Lambda^+_c$ baryon was reconstructed in the decay mode $p K^- \pi^+$. The structure is consistent with originating from a weakly decaying particle, identified as the doubly charmed baryon $\Xi_{cc}^+$. Its mass was determined to be $3261 \ \text{MeV}$.

As already noticed, many more observations in the near future are expected and their importance resides, other than in the enriching of our knowledge of hadron spectroscopy \emph{per se}, in the hope that they could shed light on the nature of strong interactions.

% vim: ft=tex nonumber wrap linebreak display+=lastline guifont=Inconsolata\ 20 spell spelllang=en_gb
