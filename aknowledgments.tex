\chapter{Ringraziamenti}

\begin{otherlanguage}{italian}
Ammetto con franchezza---e non per cospargermi il capo di cenere---che la stesura di questa tesi sarebbe stata semplicemente impossibile senza l'aiuto ed il sostegno della mia relatrice, la dottoressa Fulvia De Fazio.

Se questo può dirsi vero per qualsiasi laureando, che, prescindendo dalla bontà della sua preparazione (o piuttosto delle sue intenzioni), arriva alla soglia della laurea, in ambiti ad altissima specializzazione, appena in grado di capire di cosa si parli---fatto nei cui confronti esiste, a voler chiamare le cose col proprio nome, un tabù---, nei confronti del proprio relatore; ebbene, nel mio caso, sento di dovere alla mia relatrice particolare gratitudine, da una parte, per l'aver sopportato senza spirito di rassegnazione le mie ostinate deficienze---nonchè, a volte, la mia deficente ostinazione (nel non darle ascolto)---, dall'altra, per aver profuso, nell'opera pia di traghettare il sottoscritto a questo traguardo, un impegno eccezionale e gratuito, ed infine per le cose che ho imparato.
\end{otherlanguage}
