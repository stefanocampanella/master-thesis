\chapter{Conclusions}

Thanks to the $B$-factories and the LHCb experiment, there exist an ever growing availability of new data and hadronic spectroscopy is becoming a field of precision physics. The theoretical analysis of this data requires an approach deeply rooted in QCD. In the case of heavy-light hadrons, at present, one of the best options is the heavy chiral perturbation theory, which, unlike potential models, allows a model independent description of such states in the limit in which heavy quarks have infinite mass and light ones are massless. This theory is an established research tool and is today widely used in the discussion heavy-light hadronic systems.

At present, the states of the open-charm meson spectrum fit rather nicely the predictions of the heavy chiral perturbation theory, although the puzzle of the low mass of the strange $S$ doublet with $n = 1$ still needs to be clarified. 

%Also, several empirical mass relations have revealed themselves to be good heuristics in the discussion of the classification of states. These relations do not derive from this theory in a formal way, but in light of heavy chiral perturbation theory appears reasonable. Arguments based on them suggested that one between the $D^*_1(2680)$ and the $D^*_{s 1}(2860)$ could be a state with $n = 3$, which, if confirmed, would be the first state of this kind. However, it should be noticed that confirmation that the $D^*_1(2680)$ is really not the $D^*(2600)$ is needed in the first place. 

Classification is the first step in each science, because it allows to find patterns and indicates where to look for missing pieces. Prominent examples are the Mendeleev's table and the early quark model. In the case of heavy meson spectroscopy and in particular of open-charm spectroscopy, there are many vacancies in the classification of states and future observations will surely enrich our understanding. In particular, it would be interesting to observe the spin partner of the $D^*_{s 1}(2700)$ and in general other states with $n = 2$, indeed a glance at the table \ref{tab:charm_taxonomy} reveals that practically all radial excited states have uncertain classification. These and other considerations show how the heavy-meson spectroscopy is still a young and promising field of research.

The case of $D^*_2(3000)$ was considered in some detail and original results, based on the formalism treated in this thesis, have been presented. This meson has been observed by LHCb Collaboration in 2016, and its spin parity has been fixed to $J^P=2^+$. Considering  the classification scheme discussed in this thesis, I have found that there are  two possible identifications for it. In order to distinguish between the two, several strategies have been proposed in this thesis. 

In particular, the first one relies on the calculation of the ratio of the branching fractions of the $D^*_2(3000)$ in $D \pi$ and $D^* \pi$. This quantity is sensitive to the quantum numbers of $D^*_2(3000)$ and hence suitable to classify it. This result is interesting in light of the upcoming measurements of the LHCb collaboration.

Other original results presented in this thesis are the predictions concerning the spin and the strange partners of the $D^*_2(3000)$. In particular, I have shown that exist a hierarchy among the possible decay modes of the latter that depends on the identification of the $D^*_2(3000)$. Hence, a measurement of such hierarchy could again provide a useful tool for the identification of this meson.

Finally, I would like to mention possible future investigations stemming from this work or connected to its theoretical framework. To begin with, within the heavy mesons spectroscopy, an interesting possibility is the determination of the low energy coupling constants of the heavy chiral perturbation theory from experimental data. This requires the evaluation of decay widths to final states with a light vector meson and involves extensions of the methods presented in this thesis. On the other hand, within the latter, it is possible to calculate decays to final states with an excited meson belonging to other doublets than the fundamental one plus a light pseudoscalar meson.

The spectroscopy of baryons containing a heavy quark, which as noticed in the first chapter is collecting new experimental achievements as well, can be studied using the same methods presented in this thesis. Indeed, once the covariant representation of baryon doublets is introduced, the decay widths of excited heavy baryons to a final state with a light pseudoscalar meson can be obtained from effective Lagrangians in a similar way.

% vim: ft=tex nonumber wrap linebreak display+=lastline guifont=Inconsolata\ 20 spell spelllang=en_gb
