\chapter{Abstract}

One of the frontiers of Physics is the understanding of strong interactions, which is a challenging subject both theoretically and experimentally. Indeed, it is a remarkable achievement how the theoretical research has been able so far to extract many pieces of information from the experimental evidence, which under many aspects is very limited. On the other hand, a great deal of efforts has been put in this area of experimental research during the last decades and, as more data become available, there is hope that it will be possible to shed light on this subject, just like atomic spectroscopy was among the propellants of the quantum revolution of the 20th century.

One of these theoretical challenges is the following. Because of its success in the description of many aspects of hadronic interactions, the quantum chromodynamics is believed to be the theory behind them. However, it is unclear how to describe relativistic bound states in a quantum field theory in general and in quantum chromodynamics in particular. In other words, there are no analytic methods to calculate the spectrum of a given hadronic system.

However, as it will be shown in this thesis, mesons composed by a heavy quark and a light antiquark (\emph{heavy-light} mesons) have peculiar features that allow to describe them using approximations within the quantum chromodynamics that can be elegantly cast in the form of effective theories. One of the applications of these theories is the classification of charmed mesons, the subject of this thesis, which is an important part of the endeavour of theoretical research in hadron spectroscopy. 

\subsection{Thesis Outline}

The thesis is organized as follows:

\begin{itemize}
  \item In the first chapter I give an overview of the recent experimental results on the subject of heavy meson spectroscopy, focusing on the open-charm sector, which has seen many improvements in the last years. I also mention other important findings in QCD exotica and in the baryonic sector, which are essential to get the whole picture of the present state of our understanding of the hadronic structure. In this regard, it should be stressed the importance of ordinary hadronic spectroscopy in order to individuate or exclude possible exotic candidates.
  \item In the second chapter I overview the main features of the effective theories used in the study performed in this thesis, i.e. the heavy quark effective theory and the chiral perturbation theory. Such theories leverage one of the most beautiful and ubiquitous concepts of modern theoretical physics, the notion of symmetry. These effective theories emerge, respectively in the limit in which heavy quarks have infinite mass and light ones are massless, exploiting the symmetries that emerge in such limits. Combining the two approaches allows to make calculations of quantities which are sensitive to the quantum numbers of heavy-light hadrons and thus are suitable to classify such states. The predictions obtained in this framework are very sound, since they do not rely on approximate models, such as, for example, potential models. Instead, they stem from a theory that is obtained from quantum chromodynamics in a well defined limit (large mass of the heavy quarks, zero mass for the light quarks).
  \item In the third chapter all observed open-charm mesons are classified according to this theory. The classification of non-established states is discussed, presenting arguments based on their quantum numbers, masses and widths. Finally, it is presented an original contribution aiming at the identification of the $D^*_2(3000)$, an excited $D$ meson recently observed by the LHCb collaboration.
\end{itemize}

% vim: ft=tex nonumber wrap linebreak display+=lastline guifont=Inconsolata\ 20 spell spelllang=en_gb
