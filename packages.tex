\usepackage[hidelinks,plainpages=false,pdfpagelabels,pdfauthor={Stefano Campanella},pdftitle={Spectroscopy of Charmed Hadrons: Facing the Latest Experimental Results with the Theory}]{hyperref}

\usepackage{mathtools,amssymb}%,amsthm}
\usepackage{unicode-math}
\setmainfont{STIX Two Text}
\setmathfont{STIX Two Math}
\usepackage{microtype}

\usepackage[english]{babel}
\usepackage[a4paper]{geometry}

\newcommand{\HRule}{\rule{\linewidth}{0.5mm}}
\newlength\oldparindent\oldparindent=\parindent

% settings per fronteretro ---> IMPORTANTE: ABILITARE PAGESTYLE IN DOCUMENT.TEX
%%<--------
\makeatletter % necessary for using \@chapapp
\newcommand{\correctchaptername}{\@chapapp}
\makeatother
\renewcommand{\chaptermark}[1]{\markboth{#1}{}}
\makepagestyle{default}
\makeevenhead{default}{\textsc{\bfseries \correctchaptername{} \thechapter}}{}{}
\makeoddhead{default}{}{}{\textsc{\bfseries \leftmark}}
\makeevenfoot{default}{\bfseries \thepage}{}{}
\makeoddfoot{default}{}{}{\bfseries \thepage}
%-------->

\captionnamefont{\bfseries}

% Può risolvere alcuni warning in compilazione
%\setlength{\headheight}{14pt}

% Per rimuovere indentazione usa:
%\setlength\parindent{0pt}

% Integrazioni (riferimenti, didascalia e link)
%\usepackage[italian]{varioref}

% Unità di misura (siunits obsoleto)
%\usepackage[range-phrase={\,--\,}]{siunitx}

% Tabelle (delimitatori aggiuntivi e ambienti multicolonna)
\usepackage{booktabs}

%\newtheorem{theorem}{Teorema}[chapter]
%\newtheorem{lemma}[theorem]{Lemma}
%\newtheorem{corollary}[theorem]{Corollario}
%\theoremstyle{definition}
%\newtheorem{definition}{Definizione}[chapter]
%\newtheorem{example}{Esempio}[chapter]
%\newtheorem{xca}{Esercizio}[chapter]
%\theoremstyle{remark}
%\newtheorem{remark}{Osservazione}[chapter]

%\usepackage[caption = false]{subfig}

% Forma convezionale corretta di parte reale e parte immaginaria
\renewcommand{\Re}{\mathop{\mathrm{Re}}}
\renewcommand{\Im}{\mathop{\mathrm{Im}}}
% Forma convenzionale corretta di d (differenziale)
\renewcommand{\d}[1]{\ensuremath{\operatorname{d}\!{#1}}}
% definizione convenzionale corretta per differenziali di ordine n
\newcommand{\dn}[2][]{\ensuremath{\operatorname{d^#1}\!{#2}}}
% Derivata totale
\newcommand{\dt}[2][]{\frac{\d{#1}}{\d{#2}}}
% Derivata parziale
\newcommand{\dpar}[2][]{\frac{\partial#1}{\partial#2}}
