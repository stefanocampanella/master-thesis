\documentclass{article}

\usepackage[T1]{fontenc}
\usepackage[utf8]{inputenc}
\usepackage[a4paper]{geometry}
\usepackage{microtype}
\usepackage[italian]{babel}
\usepackage{hyphenat}
\usepackage{mathtools,amssymb,slashed,tensor}
\usepackage{booktabs}

% Adjoint
\newcommand{\adj}[1]{\overline{#1}}
% Average
\newcommand{\average}[1]{\overline{#1}}
% Trace operator
\DeclareMathOperator{\tr}{Tr}
% Differential operator
\DeclareMathOperator{\D}{d}

\begin{document}

\section{Osservazioni Preliminari}

Il punto di partenza è la definizione di simmetria in meccanica quantistica. Sia $\hat{U}$ un operatore unitario o anti-unitario che fornisce una rappresentazione infinito dimensionale (agente sullo spazio di Hilbert degli stati) di un gruppo di simmetria astratto ($SO(1,3)$, $SU(3)_C$, etc.); allora una teoria con Hamiltoniano $\hat{H}$ è invariante se $\hat{U}$ annichilisce lo stato di vuoto e
\begin{equation}
  [\hat{U},\hat{H}] = 0 \, .
\end{equation}

La precedente condizione si può riscrivere nel modo seguente
\begin{equation}
  \int \! \mathop{d^3 \! x} \hat{U} \left( : \! \dot{\hat{\phi}}(x) \hat{\pi}(x) - \mathcal{L}(\hat{\phi}(x), \hat{\pi}(x)) \! : \right) \hat{U}^{-1}  = \int \! \mathop{d^3 \! x'} \left\vert \frac{\partial x'}{\partial x} \right\vert^{-1} : \! \dot{\hat{\phi}}(x') \hat{\pi}(x') - \mathcal{L}(\hat{\phi}(x'), \hat{\pi}(x')) \! : \, .
\end{equation}
Tuttavia, in pratica, verifichiamo che
\begin{equation}
  \hat{U} \hat{\mathcal{L}}_\text{I}(\hat{\phi}(x),\partial \hat{\phi}(x)) \hat{U}^{-1} = \hat{\mathcal{L}}_\text{I}(\hat{\phi}(x'),\partial \hat{\phi}(x')) \, ,
\end{equation}
cioè che il termine di interazione della (densità di) Lagrangiana sia uno scalare.

L'azione di $\hat{U}$ sui campi della teoria quantizzata è prescritta in analogia con il caso classico: se in quest'ultimo la legge di trasformazione è $\phi'_i (x') = \Lambda_{i j} \phi_j (x)$, in quello quantistico si postula
\begin{equation}
  \langle \beta' \vert \hat{\phi}_i (x') \vert \alpha' \rangle = \langle \beta \vert \Lambda_{i j} \hat{\phi}_j (x) \vert \alpha \rangle \, ,
  \label{eq:quantum_transformation_rule}
\end{equation}
dove $\vert \alpha'/\beta' \rangle = \hat{U} \vert \alpha/\beta \rangle$ ed $\hat{U}$ è lineare. Da questa si ricava
\begin{equation}
  \hat{U} \hat{\phi}_i (x) \hat{U}^{-1} = \Lambda^{-1}_{i j} \hat{\phi}_j(x')
\end{equation}

Fanno eccezione le inversioni temporali $\hat{T}$: in questo caso gli stati iniziali e finali si scambiano e la trasformazione nel caso classico ha la forma $\phi'(t', \mathbf{x}') = T_0 K \phi(-t, \mathbf{x})$, dove $T_0$ è lineare e $K$ è l'operatore di coniugazione di complessa; pertanto la relazione postulata precedentemente va modificata nel modo seguente
\begin{equation}
  \langle \alpha' \vert \hat{\phi}(t', \mathbf{x}') \vert \beta' \rangle = \langle \beta \vert T_0 \hat{\phi}^\dagger(-t, \mathbf{x}) \vert \alpha \rangle \, , \\
\end{equation}
da cui segue
\begin{equation}
  \hat{T} \hat{\phi}(t,\mathbf{x}) \hat{T}^{-1} = \left( T_0^\dagger \right)^{-1} \hat{\phi}(-t, \mathbf{x}) \, . 
\end{equation}
con $\hat{T}$ anti-unitario (quindi anti-lineare). 

Bisogna tener presente che, dal momento che $\hat{U}$ agisce sullo spazio di Hilbert degli stati, esso commuta con gli indici spinoriali e tensoriali (es. $\hat{U} \slashed{v} \hat{\phi} \hat{U}^{-1} = \slashed{v} \hat{U}^{-1} \hat{\phi} \hat{U}$); tuttavia, se esso è anti-lineare, allora $\hat{U} a \hat{\phi} \hat{U}^{-1} = a^* \hat{U} \hat{\phi} \hat{U}^{-1}$.

Nel nostro caso, $\mathcal{L}_\text{I}$ descrive interazioni tra un generico doppietto pesante $T$, il doppietto fondamentale $H$ ed i mesoni vettoriali leggeri combinati nella matrice $\rho$. A questo punto, conviene scrivere la sua espressione più generale all'ordine $0$ in $1/m$\footnote{Più precisamente di ordine $0$ sia in $1/m_H$ che $1/m_T$} compatibile con le simmetrie di Lorentz\footnotemark{} \cite{article:Georgi}, interne e di quark pesante
\begin{multline}
  \hat{\mathcal{L}}^{(m)}_\text{I} = \int \! \mathop{d^4 v} \, \delta(v^2 -1) \theta(v_0) \frac{g_m}{\Lambda^m} \sum_{h,l,l'} \sum_{i,j,k} \left( \left. \adj{\hat{H}}(x; v) \right.^{i j}_{l h}  \left. \hat{T}^{\mu_1 \dots \mu_n}(x; v) \right.^{j k}_{h l'} \left. \Gamma\indices{_{\mu_1 \dots \mu_n}^{\nu_1 \dots \nu_m \alpha}}(v) \right.^{k i} \right. \\ \left. - \left. \adj{\hat{T}}^{\mu_1 \dots \mu_n}(x; -v) \right.^{i j}_{l h} \left. \hat{H}(x; -v) \right.^{j k}_{h l'} \left. \Gamma\indices{_{\mu_1 \dots \mu_n}^{\nu_1 \dots \nu_m \alpha}}(-v) \right.^{k i} \right) \left( i \hat{D}_{\nu_1} \dots i \hat{D}_{\nu_m} \hat{\mathcal{R}}_\alpha(x) \right)_{l' l} + (\text{ h.c. }) \, ,
  \label{eq:most_general_int_lagrangian}
\end{multline}
dove $\Lambda$ è un parametro con dimensione $+1$, $\mathcal{R}_\alpha$ è la combinazione covariante $i \frac{g_V}{\sqrt{2}} \rho_\alpha - V_\alpha$ ($V$ è la corrente vettoriale associata alla simmetria globale di sapore leggero) e $h, (l,l'), (i,j,k)$ sono indici rispettivamente di sapore pesante, leggero e spinoriali. 
\footnotetext{Tuttavia, non è chiaro come una Lagrangiana nella forma 
\begin{equation}
  \mathcal{L} = \int \! \frac{d^3 v}{2 v_0} \mathcal{L}_v \, ,
  \label{eq:georgi_lagrangian}
\end{equation}
date le relazioni di commutazione (10) in \cite{article:Georgi} e considerati stati di campi pesanti con velocità fissata, possa fornire elementi di matrice di transizione finiti. Infatti, affinché questi siano finiti, il valore medio tra lo stato iniziale e finale di $\mathcal{L}_v$ deve essere singolare, a causa della integrazione sulle velocità.}

Si fanno le seguenti osservazioni
\begin{itemize}
  \item $\mathbf{v}$ appare come un indice di somma continuo o parametro, dunque, fermo restando la sua interpretazione cinematica, non si trasforma.
  \item A fissato $m$ possono corrispondere più strutture $\Gamma$, dunque più termini di interazione, ciascuno con la sua costante di accoppiamento.
  \item Nel caso classico, l'azione (dunque la Lagrangiana) è una quantità reale, il che comporta che $\hat{\mathcal{L}}$ deve essere Hermitiano e la condizione addizionale
    \begin{equation}
     \gamma^0 \left( \Gamma\indices{_{\mu_1 \dots \mu_n}^{\nu_1 \dots \nu_m \alpha}}(v^\beta) \right)^\dagger \gamma^0 = \Gamma\indices{_{\mu_1 \dots \mu_n}^{\nu_1 \dots \nu_m \alpha}}(v^\beta) \, . 
     \label{eq:herm_constraint}
    \end{equation}
  \item Se $\Gamma$ avesse indici di sapore romperebbe le simmetrie interne: il rovescio di questo fatto è che solo le simmetrie di ri\hyp{}parametrizzazione e dello spaziotempo (sia continue che discrete) impongono vincoli utili per la determinazione di $\Gamma$.
\end{itemize}

Per poter ricavare questi vincoli, è necessario conoscere le leggi di trasformazione di $\hat{H}$, $\hat{T}$ e $\hat{\mathcal{R}}$, le quali sono determinate dalla relazione \eqref{eq:quantum_transformation_rule}. Si considerino le trasformazioni del gruppo di Lorentz. Si ha che $\mathcal{R}$ è un vettore, pertanto 
\begin{equation}
  \hat{\Lambda} \hat{\mathcal{R}} (x) \hat{\Lambda}^{-1} = \Lambda^\top \hat{\mathcal{R}}(\Lambda x) \, ,
\end{equation}
e analogamente si può mostrare che
\begin{equation}
  \hat{\Lambda} \left( i D_{\nu_1} \dots i D_{\nu_m} \hat{\mathcal{R}}_\alpha (x) \right) \hat{\Lambda}^{-1} = \Lambda\indices{^{\nu'_1}_{\nu_1}} \dots \Lambda\indices{^{\nu'_m}_{\nu_m}} \Lambda\indices{^{\alpha'}_{\alpha}} \left( i D_{\nu'_1} \dots i D_{\nu'_m} \hat{\mathcal{R}}_{\alpha'} \right) (\Lambda x)
\end{equation}
invece su $H$ e $T$ agisce il prodotto tensoriale delle rappresentazioni rispettivamente spinoriale ed anti-spinoriale (duale della spinoriale), spinoriale e anti-tensoriale-spinoriale. Dalla definizione
\begin{equation}
  H_{h l}(x; v) = \frac{1 + \slashed{v}}{2} \psi_h (x) \adj{\psi}_l (x) \frac{1 - \slashed{v}}{2} \, ,
\end{equation}
si ricava
\begin{equation}
  \hat{\Lambda} \hat{H}_{h l}(x; v) \hat{\Lambda}^{-1} = \adj{\Lambda}_{\frac{1}{2}} \hat{H}_{h l}(\Lambda x; \Lambda v) \Lambda_{\frac{1}{2}} \, ,
\end{equation}
tenendo presente che
\begin{align}
  \hat{\Lambda} \hat{\psi}(x) \hat{\Lambda}^{-1} &= \adj{\Lambda}_{\frac{1}{2}} \hat{\psi}(\Lambda x) \, , \\
  \Lambda_{\frac{1}{2}} \gamma^\mu \adj{\Lambda}_{\frac{1}{2}} &= \left( \Lambda^\top \gamma \right)^\mu \, .
\end{align}
Per $T$, definito come
\begin{equation}
  T_{h l}^{\mu_1 \dots \mu_n} (x; v) = \frac{1 + \slashed{v}}{2} \psi_h (x) \adj{A}^{\mu'_1 \dots \mu'_n}_l (x) \frac{1 - \slashed{v}}{2} \left(\eta\indices{^{\mu_1}_{\mu'_1}} - v^{\mu_1}v_{\mu'_1} \right) \dots \left(\eta\indices{^{\mu_n}_{\mu'_n}} - v^{\mu_n}v_{\mu'_n} \right)\, ,
\end{equation}
il calcolo è analogo e si ottiene
\begin{equation}
  \hat{\Lambda} \hat{T}^{\mu_1 \dots \mu_n}_{l h}(x; v) \hat{\Lambda}^{-1} = \Lambda\indices{_{\mu'_1}^{\mu_1}} \dots \Lambda\indices{_{\mu'_n}^{\mu_n}} \adj{\Lambda}_{\frac{1}{2}} \hat{T}^{\mu'_1 \dots \mu'_n}_{l h}(\Lambda x; \Lambda v ) \Lambda_{\frac{1}{2}} \, ,
\end{equation}
tenendo presente che
\begin{equation}
  \left(\eta\indices{^\mu_\nu} - v^\mu v_\nu \right) \left( \Lambda^\top \right)\indices{^\nu_\alpha} = \left( \Lambda^\top \right)\indices{^\mu_\nu} \left(\eta\indices{^\nu_\alpha} - v^{\prime \nu} v'_\alpha \right) \text{ con } v' = \Lambda v \, .
\end{equation}
Bisogna notare che, come effetto combinato della presenza dei proiettori di energia e delle leggi di trasformazione, i campi trasformati di $H$ e $T$ sono funzione dei vecchi campi calcolati con il nuovo valore $\mathbf{v}'$ del parametro $\mathbf{v}$.

Pertanto $\Gamma$ deve soddisfare la seguente
\begin{equation}
  \Lambda\indices{_{\mu'_1}^{\mu_1}} \dots \Lambda\indices{_{\mu'_n}^{\mu_n}} \Lambda\indices{^{\nu'_1}_{\nu_1}} \dots \Lambda\indices{^{\nu'_m}_{\nu_m}} \Lambda\indices{^{\alpha'}_{\alpha}}   \Lambda_{\frac{1}{2}} \Gamma\indices{_{\mu_1 \dots \mu_n}^{\nu_1 \dots \nu_m \alpha}}(v)  \adj{\Lambda}_{\frac{1}{2}}  =  \Gamma\indices{_{\mu'_1 \dots \mu'_n}^{\nu'_1 \dots \nu'_m \alpha'}}(\Lambda v)
  \label{eq:lorentz_constraint}
\end{equation} 

Le leggi per trasformazioni discrete si ricavano in maniera analoga, partendo da quelle per uno spinore \cite{book:Greiner}
\begin{align}
  \hat{C} \hat{\psi}(t, \mathbf{x}) \hat{C}^{-1} &= C_0 \adj{\hat{\psi}}^{\top}(t, \mathbf{x}) \quad \text{ con } C_0 = i \gamma^0 \gamma^2  \, ,\\
  \hat{P} \hat{\psi}(t, \mathbf{x}) \hat{P}^{-1} &= P_0 \hat{\psi}(t, -\mathbf{x}) \quad \text{ con } P_0 = \gamma^0 \, , \\
  \hat{T} \hat{\psi}(t, \mathbf{x}) \hat{T}^{-1} &= T_0 \hat{\psi}(-t, \mathbf{x}) \quad \text{ con } T_0 = i \gamma^1 \gamma^3  \, .
\end{align}
Da queste si ricavano le leggi di trasformazione per $\hat{H}$
\begin{align}
  \hat{C} \hat{H}(x^\alpha; v^\beta) \hat{C}^{-1} &= C_0 \adj{\hat{H}}^{\top}(x^\alpha; -v^\beta) C_0 \, ,\\
  \hat{P} \hat{H}(x^\alpha; v^\beta) \hat{P}^{-1} &= P_0 \hat{H}(x_\alpha; v_\beta) P_0 \, , \\
  \hat{T} \hat{H}(x^\alpha; v^\beta) \hat{T}^{-1} &= T_0 \hat{H}(-x_\alpha; v_\beta) T_0 \, ,
\end{align}
e per $\hat{T}$
\begin{align}
  \hat{C} \hat{T}^{\mu_1 \dots \mu_n}(x^\alpha; v^\beta) \hat{C}^{-1} &= C_0 \adj{\hat{T}}^{\top \mu_1 \dots \mu_n}(x^\alpha; -v^\beta) C_0 \, ,\\
  \hat{P} \hat{T}^{\mu_1 \dots \mu_n}(x^\alpha; v^\beta) \hat{P}^{-1} &= P_0 \hat{T}_{\mu_1 \dots \mu_n}(x_\alpha; v_\beta) P_0 \, , \\
  \hat{T} \hat{T}^{\mu_1 \dots \mu_n}(x^\alpha; v^\beta) \hat{T}^{-1} &= T_0 \hat{T}_{\mu_1 \dots \mu_n}(-x_\alpha; v_\beta) T_0 \, .
\end{align}

Nel caso della $\rho$, la legge di trasformazione per parità è quella per un vettore
\begin{equation}
  \hat{P} \hat{\rho}^\mu(x^\alpha) \hat{P}^{-1} = \hat{\rho}_\mu(x_\alpha) \, , 
\end{equation}
e lo stesso vale per la inversione temporale
\begin{equation}
  \hat{T} \hat{\rho}^\mu(x^\alpha) \hat{T}^{-1} = \hat{\rho}_\mu(-x_\alpha) \, ,
\end{equation}
tenendo sempre presente che, essendo $T$ anti-lineare, quando si valuta $\hat{T}^{-1} \hat{\mathcal{L}}_\text{I}(\hat{H}, \hat{T}, \hat{\rho}) \hat{T}$ bisogna complesso coniugare tutti i coefficienti numerici che compaiono nella espressione di $\hat{\mathcal{L}}_\text{I}$.

Nel caso della coniugazione di carica, bisogna anche trasporre $\rho$ \cite{article:Tyutin}, che ha due indici di sapore leggero, in modo da scambiare i mesoni vettoriali con le loro antiparticelle (i mesoni sulla diagonale, $\rho^0$, $\phi$ e $\omega$, cambiano solo segno, in accordo al fatto che sono autostati di carica con autovalore $-1$)
\begin{equation}
  \hat{C} \hat{\rho}^\mu(x^\alpha) \hat{C}^{-1} = - \hat{\rho}^{\top \mu}(x^\alpha) \, .
\end{equation}

Infine, rimangono da chiarire due punti. Il primo è se $V$ abbia le stesse proprietà di trasformazione di $\rho$ (in particolare per coniugazione di carica), in modo che la definizione di $\mathcal{R}$ abbia senso. Il secondo è capire come si trasformi $i D^{\nu_1} \dots i D^{\nu_m} \mathcal{R}^\alpha$, dove $D$ è la derivata covariante che agisce su un oggetto covariante $X$ con due indici di sapore leggero (uno sede della rappresentazione fondamentale ed uno della sua duale)
\begin{equation}
  D_\mu X = \partial_\mu X + \frac{1}{2} \left[i \frac{g_V}{\sqrt{2}} \rho_\mu + V_\mu, X \right] \, .
  \label{eq:covariant_derivative}
\end{equation}

Si tralascia la verifica del primo punto e si assume per buona la definizione di $\mathcal{R}$. Circa il secondo, si osserva che la combinazione $i \partial_\mu$ si trasforma come un vettore per trasformazioni sia continue che discrete, dunque termini del tipo $i \partial^{\nu_1} \dots i \partial^{\nu_m} \mathcal{R}^\alpha$ si trasformano come un tensore. Si può verificare per ricorsione che il termine in cui appare il commutatore in \eqref{eq:covariant_derivative} ha le corrette proprietà di trasformazione, pertanto
\begin{align}
  \hat{C} \left( i \hat{D}^{\nu_1} \dots i \hat{D}^{\nu_m} \hat{\mathcal{R}}^\alpha (t, \mathbf{x}) \right) \hat{C}^{-1} &= - \left( i \hat{D}^{\nu_1} \dots i \hat{D}^{\nu_m} \hat{\mathcal{R}}^\alpha (t, \mathbf{x}) \right)^\top \, ,\\
  \hat{P} \left( i \hat{D}^{\nu_1} \dots i \hat{D}^{\nu_m} \hat{\mathcal{R}}^\alpha (t, \mathbf{x}) \right) \hat{P}^{-1} &= i \hat{D}_{\nu_1} \dots i \hat{D}_{\nu_m} \hat{\mathcal{R}}_\alpha (t, -\mathbf{x}) \, ,\\
  \hat{T} \left( i \hat{D}^{\nu_1} \dots i \hat{D}^{\nu_m} \hat{\mathcal{R}}^\alpha (t, \mathbf{x}) \right) \hat{T}^{-1} &= i \hat{D}_{\nu_1} \dots i \hat{D}_{\nu_m} \hat{\mathcal{R}}_\alpha (-t, \mathbf{x}) \, ,\\
\end{align}

A questo punto è necessario fare una precisazione. Se nella \eqref{eq:most_general_int_lagrangian} fosse mancato il secondo termine della corrente di doppietto pesante (quello in cui compare $-v$), non sarebbe stato possibile ottenere invarianza per coniugazione di carica (come segnalato dalle leggi di trasformazione di $H$ e $T$ e dal fattore $-1$ associato a $\mathcal{R}_c$). Questo fatto ha un analogo anche in elettrodinamica, dove è necessario utilizzare la espressione anti-simmetrizzata della corrente fermionica \cite{book:Greiner} (p. 315)
\begin{equation}
  j^\mu = \frac{e}{2} \left[ \, \adj{\psi}, \gamma^\mu \psi \right] \, .
  \label{eq:antisymmetrized_current}
\end{equation}
Infatti si trova che
\begin{equation}
  \hat{C} \adj{\hat{\psi}} \gamma^\mu \hat{\psi} \hat{C}^{-1} = \hat{\psi}^\top \gamma^{\mu \top} \hat{\adj{\psi}} \vphantom{\psi}^\top \neq - \adj{\hat{\psi}} \gamma^\mu \hat{\psi} \, ,
\end{equation}
mentre usando la \eqref{eq:antisymmetrized_current} si trova il risultato corretto. Detto questo, in QED non è necessario fare esplicitamente uso della espressione anti-simmetrizzata in quanto si trova che \cite{book:Greiner} (pp. 134-135)
\begin{equation}
  \hat{C} :\adj{\hat{\psi}} \gamma^\mu \hat{\psi}: \hat{C}^{-1} = - : \adj{\hat{\psi}} \gamma^\mu \hat{\psi} :  
\end{equation}
e, dal momento che lo Hamiltoniano è definito con prescrizione di ordinamento normale, si ottiene il risultato corretto $[\hat{C}, \hat{H}] = 0$. 

Tuttavia, la prescrizione di ordinamento normale non è utile nel caso della Lagrangiana effettiva in considerazione. Infatti, per effetto dei proiettori di energia nella loro definizione, $\hat{H}(x;v)$ e $\hat{T}(x;v)$ contengono solo gradi di libertà (cioè operatori di creazione e distruzione) di particella e $\hat{H}(x;-v)$ e $\hat{T}(x;-v)$ di antiparticella. Dal momento che l'operatore di coniugazione di carica scambia particelle con antiparticelle, devono apparire entrambi i campi (cioè con $v$ e $-v$) in \eqref{eq:most_general_int_lagrangian} affinché la Lagrangiana sia uno scalare. In altri termini, bisogna utilizzare esplicitamente la corrente di doppietto pesante anti-simmetrizzata. Da un punto di vista intuitivo,  l'espressione anti-simmetrizzata in \eqref{eq:most_general_int_lagrangian} corrisponde nella analogia con la QED a considerare sia i contributi di corrente elettronica che positronica (con carica opposta). Invece, si osserva che l'inclusione dello Hermitiano coniugato in \eqref{eq:most_general_int_lagrangian} (richiesta dalla Hermitianità della Lagrangiana) non è coinvolta nella simmetria di carica; infatti, scambia operatori di creazione e distruzione associati agli stessi gradi di libertà e permette di considerare processi in cui $H$ è nello stato iniziale e $T$ nello stato finale e viceversa.

In virtù delle leggi di trasformazione dei campi della teoria, si trovano le seguenti
\begin{subequations}
\begin{align}
  C_0 \Gamma\indices{_{\mu_1 \dots \mu_n}^{\nu_1 \dots \nu_m \alpha}}(v^\beta) C_0 &= \left( \Gamma\indices{_{\mu_1 \dots \mu_n}^{\nu_1 \dots \nu_m \alpha}}(-v^\beta) \right)^\top \, , \\
  P_0 \Gamma\indices{_{\mu_1 \dots \mu_n}^{\nu_1 \dots \nu_m \alpha}}(v^\beta) P_0 &= \Gamma\indices{^{\mu_1 \dots \mu_n}_{\nu_1 \dots \nu_m \alpha}}(v_\beta) \, , \\
  T_0 \left( \Gamma\indices{_{\mu_1 \dots \mu_n}^{\nu_1 \dots \nu_m \alpha}}(v^\beta) \right)^* T_0 &= \Gamma\indices{^{\mu_1 \dots \mu_n}_{\nu_1 \dots \nu_m \alpha}}(v_\beta) \, .
\end{align}
\label{eq:CPT}
\end{subequations}

A questo punto è necessario considerare l'invarianza di ri\hyp{}parametrizzazione. Questa simmetria è associata alla parametrizzazione dei doppietti pesanti di momento $p^\mu$ in arbitrarie velocità fissata $\mathbf{v}$ e momento residuo $\mathbf{k}$, in modo che $p = m v + k$ ($m$ massa del doppietto pesante). La seguente osservazione è di fondamentale importanza: le componenti $0$ dei quadrivettori $v$ e $k$ sono positive e fissate dalle condizioni
\begin{align}
  v^2 &= 1 \, , \\
  p^2 &= m^2 \, ,
\end{align}
da cui segue
\begin{align}
  v_0 &= \sqrt{1 + \Vert \mathbf{v} \Vert^2} \, , \\
  \frac{k_0}{m} &= \sqrt{v_0^2 + 2 \left\Vert \mathbf{v} + \frac{\mathbf{k}}{m} \right\Vert^2 + \left\Vert \frac{\mathbf{k}}{m} \right\Vert^2} - v_0^2 \, .
\end{align}
I parametri $v$ e $k$ si trasformano sotto ri\hyp{}parametrizzazione di una quantità $\mathbf{q}$ nel modo seguente:
\begin{equation}
  (v, k) \to (v + \frac{q}{m}, k - q)  \, ,
\end{equation}
dove la componente $0$ di $q$ è fissata dalle condizioni precedenti, dunque
\begin{equation}
\left( v + \frac{q}{m} \right)^2 = 1 \, .
\end{equation}
Si osserva che, se le trasformazioni di ri\hyp{}parametrizzazione fossero definite in termini del generico quadrivettore $q$ invece che del vettore $\mathbf{q}$, la condizione $(v - q/m)^2 = 1$, soddisfatta per ogni $v$, implicherebbe $q = 0$.

Si riassume brevemente il ragionamento in \cite{article:Luke_Manohar} per un campo pesante scalare $\phi_v$. La Lagrangiana più generale possibile\footnote{Notare che, in accordo alla precisazione precedente, la somma è sui vettori $\mathbf{v}$.} 
\begin{equation}
  \mathcal{L} = \sum_\mathbf{v} \mathcal{L} \left( \phi_v (x), v^\mu, i D^\mu \right) \, ,
  \label{eq:luke_manohar_sum}
\end{equation}
si trasforma sotto ri\hyp{}parametrizzazione nel modo seguente
\begin{equation}
  \mathcal{L}' = \sum_\mathbf{v} \mathcal{L} \left( \phi_{v+q/m}, v^\mu, i D^\mu + q^\mu \right) = \sum_\mathbf{w} \mathcal{L} \left( \phi_w, w^\mu - \frac{q^\mu}{m}, i D^\mu + q^\mu \right) \, ,
\end{equation}
fatto da cui discende la prescrizione in \cite{article:Luke_Manohar} per cui $v^\mu$ e le derivate dei campi pesanti $D^\mu$ devono apparire nella combinazione covariante per trasformazioni di ri\hyp{}parametrizzazione
\begin{equation}
  \mathcal{V}^\mu = v^\mu + \frac{i D^\mu}{m} \, .
\end{equation}
Nel caso di campi pesanti con una struttura tensoriale o spinoriale, questi ultimi vengono ridefiniti in maniera opportuna utilizzando combinazioni di trasformazioni di Lorentz $\Lambda(p/m,u)$, nella opportuna rappresentazione (spinoriale o tensoriale), definite in modo da ruotare $u$ su $p/m$. In questo modo si ottiene che i campi ridefiniti (dove il momento $p$ appare tramite le derivate $i D$), per trasformazioni di ri\hyp{}parametrizzazione, acquistano semplicemente un fattore di fase. 

Detto questo, esistono alcune difficoltà nel caso in esame. Per cominciare, il fatto che i campi di doppietto pesante a velocità fissata non vengono ottenuti con una procedura \textit{ab initio} come nel caso dei quark pesanti in HQET; in altri termini non esiste,ad esempio, una Lagrangiana per $H(x)$ ed una definizione di $H(x;v)$ in termini di $H(x)$.

Nonostante questo, si può richiedere ragionevolmente che l'operatore unitario $\hat{R}(\mathbf{q}/m)$ associato alla ri\hyp{}parametrizzazione di una quantità $\mathbf{q}$ faccia corrispondere a $H(x;v)$ il campo $H(x;v+q/m)$, dunque
\begin{equation}
  \hat{R}\left(\frac{\mathbf{q}}{m}\right) \hat{H}(x;v) \hat{R}^{-1}\left(\frac{\mathbf{q}}{m}\right) = \hat{H}(x; v-q/m) 
  \label{eq:reparametrization_law}
\end{equation}
ed analogamente per il campo di anti\hyp{}doppietto $H(x; -v)$, dove in quest'ultimo caso la legge di ri\hyp{}parametrizzazione è
\begin{equation}
  (-v, k) \to (-v - \frac{q}{m}, k - q) \, .
\end{equation}
Inoltre, la relazione tra $H(x;v-q/m)$ ed $H(x;v)$ può essere ottenuta, a partire dalla definizione di $H$, usando le trasformazioni di Lorentz $\Lambda(v,p/m)$ e $\Lambda(v',p/m)$ seguendo quanto in \cite{article:Luke_Manohar}
\begin{equation}
  H\left(x;v-\frac{q}{m}\right) = e^{- i q \cdot x} \Lambda_{\frac{1}{2}}\left(v-\frac{q}{m},\frac{p}{m}\right) \Lambda_{\frac{1}{2}}\left(\frac{p}{m}, v\right) H(x;v) \Lambda_{\frac{1}{2}}\left(v,\frac{p}{m}\right) \Lambda_{\frac{1}{2}}\left(\frac{p}{m}, v - \frac{q}{m}\right) 
\end{equation}
ed applicando la sostituzione (all'ordine opportuno in $1/m$)
\begin{equation}
  \frac{p^\mu}{m} \to \frac{v^\mu + i \frac{D^\mu}{m}}{\left\Vert v^\mu + i \frac{D^\mu}{m} \right\Vert} \, .
\end{equation}
In maniera analoga può essere fatto per $T$.

Un'altra difficoltà deriva dal fatto che nella teoria che si vuole descrivere appaiono almeno due diversi doppietti di campo pesante ($T$ e $H$). Infatti, nella legge di ri\hyp{}parametrizzazione appare la massa di doppietto pesante, che sarà diversa nei due casi. Dunque, affinché la invarianza per ri\hyp{}parametrizzazione sia compatibile con la regola di super\hyp{}selezione delle velocità, non è possibile fissare indipendentemente i due momenti residui e dovrà valere
\begin{equation}
  \frac{\mathbf{q}_H}{m_H} = \frac{\mathbf{q}_T}{m_T} \, .
  \label{eq:repar_momentum_constraint}
\end{equation}

Con queste precisazioni, è possibile ridefinire i campi di doppietto pesante in modo che acquistino semplicemente una fase per trasformazioni di ri\hyp{}parametrizzazione
\begin{align}
  \mathcal{H}(x; v) &= \Lambda_{\frac{1}{2}} \left(\frac{p}{m_H}, v\right) H(x; v) \Lambda_{\frac{1}{2}} \left(v, \frac{p}{m_H}\right) \, , \\ 
  \mathcal{T}^{\mu_1 \dots \mu_n}(x; v) &= \Lambda\indices{^{\mu_1}_{\mu'_1}}\left(\frac{p}{m_H}, v\right) \dots \Lambda\indices{^{\mu_n}_{\mu'_n}}\left(\frac{p}{m_H}, v\right)  \Lambda_{\frac{1}{2}} \left(\frac{p}{m_H}, v\right) \left( T^{\mu'_1 \dots \mu'_n}(x; v) \right) \Lambda_{\frac{1}{2}} \left(v, \frac{p}{m_H}\right) \, .
\end{align}
In questo modo si può ripercorrere il ragionamento di \cite{article:Luke_Manohar} senza problemi e concludere che $\Gamma$ deve dipendere da $\mathcal{V}$ e non da $v$. Pertanto la nuova Lagrangiana sarà nella forma (utilizzando una notazione leggermente più compatta)
\begin{multline}
  \hat{\mathcal{L}}^{(m)}_\text{I} = \int \! \mathop{d^4 v} \, \delta(v^2 -1) \theta(v_0) \frac{g_m}{\Lambda^m} \tr \left( \adj{\hat{\mathcal{H}}}(x; \mathcal{V}) \hat{\mathcal{T}}^{\mu_1 \dots \mu_n}(x; \mathcal{V}) \Gamma\indices{_{\mu_1 \dots \mu_n}^{\nu_1 \dots \nu_m \alpha}}(\mathcal{V}) i \hat{D}_{\nu_1} \dots i \hat{D}_{\nu_m} \hat{\mathcal{R}}_\alpha(x) \right) \\ - \tr \left( \adj{\hat{\mathcal{H}}}(x; -\mathcal{V}) \hat{\mathcal{T}}^{\mu_1 \dots \mu_n}(x; -\mathcal{V}) \Gamma\indices{_{\mu_1 \dots \mu_n}^{\nu_1 \dots \nu_m \alpha}}(-\mathcal{V}) i \hat{D}_{\nu_1} \dots i \hat{D}_{\nu_m} \hat{\mathcal{R}}_\alpha(x) \right)  + (\text{ h.c. }) \, ,
\end{multline}

Ma a questo punto si ricorda che la Lagrangiana di interazione che si vuole costruire è di ordine $0$ in $1/m$. A quest'ordine i campi pesanti ridefiniti coincidono con le loro versioni non-covarianti per ri\hyp{}parametrizzazione
\begin{align}
  \mathcal{H}(x;v) &= H(x;v) + \mathcal{O}\left(\frac{1}{m}\right) \, , \\
  \mathcal{T}^{\mu_1 \dots \mu_n}(x;v) &= T^{\mu_1 \dots \mu_n}(x;v) + \mathcal{O}\left(\frac{1}{m}\right) \, , 
\end{align}
e la Lagrangiana più generale possibile non potrà contenere derivate di questi ultimi (che è l'ansatz fatto in partenza nella \eqref{eq:most_general_int_lagrangian}), dunque $\Gamma$ torna a dipendere solo da $v$. Il nodo della questione è che ora l'invarianza per ri\hyp{}parametrizzazione richiede che
\begin{equation}
  \Gamma\indices{_{\mu_1 \dots \mu_n}^{\nu_1 \dots \nu_m \alpha}}(v) = \Gamma\indices{_{\mu_1 \dots \mu_n}^{\nu_1 \dots \nu_m \alpha}}\left( v - \frac{q}{m} \right) = \Gamma\indices{_{\mu_1 \dots \mu_n}^{\nu_1 \dots \nu_m \alpha}}\left( v \right) + \mathcal{O}\left(\frac{1}{m}\right) \, , 
\end{equation}
dunque, all'ordine zero in $1/m$, questa relazione non fornisce un vincolo utile per la determinazione di $\Gamma$. Nonostante questo, si è preferito esaminare l'invarianza per ri\hyp{}parametrizzazione in dettaglio per verificare che il caso dei campi di doppietto pesante non costituisce una eccezione rispetto alle considerazioni in \cite{article:Luke_Manohar} ed evidenziare alcuni punti concettualmente rilevanti (es. la condizione \eqref{eq:repar_momentum_constraint}).

\section{Calcolo delle Larghezze di Decadimento}

Le equazioni \eqref{eq:herm_constraint}, \eqref{eq:lorentz_constraint} e \eqref{eq:CPT}, costituiscono i vincoli disponibili per stabilire le strutture possibili di $\Gamma$. Si sottolinea la seguente: si consideri il tensore $\Gamma$ avente $n+m+1$ indici contro\hyp{}varianti $\lambda_i$, allora è semplice mostrare che la forma più generale per \Gamma compatibile con tutti vincoli è la seguente
\begin{equation}
  \Gamma^{\lambda_1 \dots \lambda_{n+m+1}} = \chi^{\lambda_1} \dots \chi^{\lambda_{n+m+1}} \text{ con } \chi^\lambda = v^\lambda, \gamma^\lambda \, .
\end{equation}
In linea di principio sarebbe potuto apparire nella struttura $\Gamma$ il tensore metrico $\eta^{\lambda_i \lambda_j}$, tuttavia quest'ultimo coincide sia con la parte scalare di $\gamma^{\lambda_i} \gamma^{\lambda_j}$ che di $v^{\lambda_i} v^{\lambda_j}$.

Inoltre, nel calcolo, si potrà tener conto delle seguenti proprietà di $T$:
\begin{itemize}
  \item $T$ è completamente simmetrico ed a traccia nulla (ovvero è un tensore puro), 
  \item le contrazioni di un indice $\mu$ con $v$ o $\gamma$ sono nulle (trasversalità di $T$)
\end{itemize}
e di $\mathcal{R}$ (nei processi ad albero considerati):
\begin{itemize}
  \item $\mathcal{R}$ coincide con $\rho$
  \item $i D_{\nu_1} \dots i D_{\nu_m}$ coincide con $i \partial_{\nu_1} \dots i \partial_{\nu_m}$ e gli indici $\nu$ sono completamente simmetrici,
  \item le contrazioni di un indice $\nu$ con $\alpha$ sono nulle (trasversalità di $\rho$).
\end{itemize}
Comunque si osserva che, ad esempio, le proprietà di simmetria di $T^{\mu_1 \dots \mu_n}$ non costituiscono dei vincoli per $\Gamma$ ed al più permettono di escludere a priori particolari strutture che, se considerate esplicitamente, non contribuiscono alle larghezze di decadimento\footnote{Tuttavia contribuiscono nelle ampiezze e, quando questi contributi vengono saturati con i tensori di polarizzazione, essi si annullano}.

Si considera a posteriori il problema sulla necessità di includere o meno il tensore di campo di $\rho$
\begin{equation}
  F^{\mu \nu} = \partial_\mu \rho_\nu - \partial_\nu \rho_\mu + i \frac{g_V}{2} \left[ \rho_\mu , \rho_\nu \right] \, ,
\end{equation}
nella espressione della Lagrangiana di interazione più generale possibile. Questo tensore è necessariamente richiesto nel termine di pura gauge della Lagrangiana, dove fornisce il termine cinetico quadratico nelle derivate di $\rho$ ed il termine di interazione corrispondente al vertice con tre campi $\rho$. Si riporta questo termine di pura gauge citando quanto in \cite{article:Casalbuoni} ed adattandolo alla notazione qui usata
\begin{equation}
  \mathcal{L}_\rho = - \frac{1}{4} \tr \left( F^{\alpha \beta} F_{\alpha \beta} \right) - \frac{a f_\pi }{2} \tr \left( \mathcal{R}^2 \right) \, ,
\end{equation}
dove è stato incluso il termine quadratico in $\mathcal{R}$ (che contiene il termine di massa per $\rho$), fra l'altro responsabile delle relazioni KSFR.
Il tensore di campo $F$ è per definizione covariante per trasformazioni del gruppo di simmetria (nascosta) locale, come per qualsiasi altra teoria di gauge non-Abeliana. Tuttavia, diversamente da una generica teoria di gauge non-Abeliana, in quella in considerazione si ha a disposizione il campo composto $\mathcal{R}$, anch'esso covariante, oltre che lineare e privo di derivate di $\rho$. La conseguenza più eclatante di questo fatto è possibile costruire una Lagrangiana tale che il campo di gauge $\rho$ abbia massa e senza rompere l'invarianza per trasformazioni di gauge. La seconda conseguenza più rilevante è che è possibile costruire termini di interazione in cui compare il campo fondamentale $\rho$ (all'interno di $\mathcal{R}$) e le sue derivate. Anche se questo fatto non ha un analogo nelle usuali teorie di gauge non-Abeliane, è anche vero che la prescrizione di accoppiamento minimale coinvolge il campo di gauge, non il suo tensore di campo.

Queste considerazioni sono in accordo al termine di interazione della Lagrangiana (4.3) in \cite{article:Casalbuoni} in cui non compare $F$, bensì $\rho$.

Inoltre, vale la seguente considerazione: dalla relazione esatta
\begin{equation}
  i D_{\left[ \alpha \right.} \rho_{\left. \beta \right]} = - \frac{g_V}{2 \sqrt{2}} F_{\alpha \beta} + \frac{i}{2} \left( \partial_\alpha V_\beta - \partial_\beta V_\alpha +  \left[ V_\alpha , V_\beta \right] \right) \, ,
\end{equation} 
segue che, nel calcolo dei processi considerati, $F_{\alpha \beta}$ è semplicemente proporzionale alla parte antisimmetrica di $i D_\alpha \mathcal{R}_\beta$. Pertanto, se si includono termini con operatori di dimensione $m$ per $\mathcal{R}$ ed $m-1$ per $F$ (ovvero con rispettivamente $m$ ed $m-1$ derivate), si contano due volte i contributi di antisimmetrici derivanti da $i \partial \rho$.

\begin{table}
  \centering
  \begin{tabular}{ccccccc}
    \toprule
    Doppietto      & $H$ & $S$ & $T$ & $X$ & $X'$ & $F$ \\
    \midrule
    $m_\text{min}$ &  1  &  0  &  2  &  1  &  3   &  2  \\
    \bottomrule
  \end{tabular}
  \caption{Valori minimi di $m$ per ogni doppietto riferiti a strutture contenenti solo matrici $\gamma$.}
  \label{tab:m}
\end{table}

Da una verifica diretta, per avere larghezze di decadimento fisicamente consentite non nulle ad albero, bisogna considerare strutture fino ad un certo valore minimo di $m$ per ciascun doppietto (nella tabella \ref{tab:m} sono state considerate strutture contenenti solo matrici $\gamma$, pertanto questi valori sono possibilmente sovrastimati). Nel seguito si considerano i decadimenti di doppietti con $S_\ell = \frac{1}{2}$.

\paragraph{Decadimenti di $S$}
Conviene partire dal caso più semplice, ovvero per il doppietto $S$ che ha valore minimo $m = 0$. In questo caso le strutture per $\Gamma$ sono solo due ed è possibile includerle nella espressione di $\tilde{\Gamma}$ che contiene anche le costanti di accoppiamento
\begin{equation}
  \tilde{\Gamma}^\alpha = g_{0,1} v^\alpha + g_{0,2} \gamma^\alpha \, .
\end{equation}
Dalla precedente si ottengono le seguenti larghezze di decadimento, dove $\mathbf{q}$ è l'impulso del vettore leggero nello stato finale.
\begin{align}
  \Gamma\left( P^*_0 \to P^* V \right) =& \frac{m_{P^*} \Vert \mathbf{q} \Vert}{2 \pi  m_{P_0^*} m_V^2} g_{0,2}^2 \left(3 m_V^2+ \Vert \mathbf{q} \Vert^2\right) \, , \\
  \Gamma\left( P'_1 \to P V \right) =& \frac{m_P \Vert \mathbf{q} \Vert}{6 \pi  m_V^2 m_{P'_1}} g_{0,2}^2 \left(3 m_V^2+ \Vert \mathbf{q} \Vert^2\right) \, , \\
  \Gamma\left( P'_1 \to P^* V \right) =& \frac{m_{P^*} \Vert \mathbf{q} \Vert }{3 \pi  m_V^2 m_{P'_1}} g_{0,2}^2 \left(3 m_V^2+ \Vert \mathbf{q} \Vert^2\right) \, . 
\end{align}
Come si vede, in questo caso particolare, la struttura con $v$ non contribuisce.

\paragraph{Decadimenti di $\tilde{H}$}
Si passa ora al caso leggermente più complicato di $\tilde{H}$. In questo caso $m_\text{min} = 1$, pertanto le strutture necessarie ad ottenere larghezze non nulle saranno del tipo
\begin{align}
  \tilde{\Gamma}^\alpha &= g_{0,1} v^\alpha + g_{0,2} \gamma^\alpha \, , \\
  \Gamma^{\nu \alpha} &= g_{1,1} v^\nu v^\alpha + g_{1,2} \gamma^\nu v^\alpha + g_{1,3} v^\nu \gamma^\alpha + g_{1,4} \gamma^\nu \gamma^\alpha \, .
\end{align}

Utilizzando queste strutture è possibile effettuare i calcoli e si ottiene
\begin{align}
  \Gamma\left( \tilde{P} \to P V \right) &=  \frac{\Vert \mathbf{q} \Vert^3 m_P}{2 \pi \Lambda^2 m_V^2 m_{\tilde{P}}} \left(\Lambda \left(g_{0,1}-g_{0,2}\right)+\left(-g_{1,1}+g_{1,2}+g_{1,3}\right) \sqrt{m_V^2+\Vert \mathbf{q}\Vert^2}\right)^2\, , \\
  \Gamma\left( \tilde{P} \to P^* V \right) &= \frac{m_{P^*} \Vert \mathbf{q} \Vert^3}{\pi  \Lambda ^2 m_{\tilde{P}}} g_{1,4}^2 \, , \\
  \Gamma\left( \tilde{P}^* \to P V \right) &= \frac{m_P \Vert \mathbf{q} \Vert^3}{3 \pi  \Lambda ^2 m_{\tilde{P}^*}} g_{1,4}^2 \, , 
\end{align}
\begin{multline}
  \Gamma\left( \tilde{P}^* \to P^* V \right) = \frac{\Vert\mathbf{q}\Vert^3 m_{P^*}}{6 \pi \Lambda ^2 m_V^2 m_{\tilde{P}^*}} \Bigg(3 \Lambda \left(g_{0,1}-g_{0,2}\right)  \\
    \left(\Lambda \left( g_{0,1}-g_{0,2}\right)+2 \left(-g_{1,1}+g_{1,2}+g_{1,3}\right) \sqrt{m_V^2+\Vert\mathbf{q}\Vert^2}\right) \\
 + \left(3 \left(-g_{1,1}+g_{1,2}+g_{1,3}\right){}^2+4 g_{1,4}^2\right) m_V^2+3 \Vert\mathbf{q}\Vert^2 \left(-g_{1,1}+g_{1,2}+g_{1,3}\right)^2\Bigg)
\end{multline}
Dunque le due strutture con $m=0$ danno gli stessi contributi. Inoltre, circa le strutture con $m = 1$, appaiono espressioni che contengono combinazioni delle costanti di accoppiamento. Questo deriva da ridondanze e mescolamenti di strutture con proprietà di simmetria degli indici differenti. Infatti si ha che
\begin{align}
  v^\alpha v^\beta &= \left( v^\alpha v^\beta - g^{\alpha \beta} \right) + g^{\alpha \beta} \, , \\
  v^\alpha \gamma^\beta &= \frac{1}{2} \left( v^\alpha \gamma^\beta - v^\beta \gamma^\alpha \right) + \left( \frac{1}{2} \left( v^\alpha \gamma^\beta + v^\beta \gamma^\alpha \right) - \slashed{v} \eta^{\alpha \beta} \right) + \slashed{v} \eta^{\alpha \beta} \, ,\\
  \gamma^\alpha \gamma^\beta &= 2 i \sigma^{\alpha \beta} + \eta^{\alpha \beta} \, .
\end{align}
Dunque si ridefinisce $\tilde{\Gamma}^{\nu \alpha}$ in modo che le costanti di accoppiamento moltiplichino strutture con proprietà definite di simmetria ed evitando ridondanze
\begin{multline}
  \tilde{\Gamma}^{\nu \alpha} = g_{1,1} \eta^{\nu \alpha} + g_{1,2} \slashed{v} \eta^{\nu \alpha} + g_{1,3} \left( v^\alpha v^\beta - g^{\alpha \beta} \right) + g_{1,4} \frac{1}{2} \left( v^\alpha \gamma^\beta + v^\beta \gamma^\alpha \right) \\ + g_{1,5} \frac{1}{2} \left( v^\alpha \gamma^\beta - v^\beta \gamma^\alpha \right) + g_{1,6} 2 i \sigma^{\nu \alpha}  \, 
\end{multline}
dove, per comodità di notazione, sono stati usati gli stessi simboli per le costanti di accoppiamento.
Espresse in termini del nuovo $\tilde{\Gamma}$ si trovano le seguenti larghezze
\begin{align}
  \Gamma\left( \tilde{P} \to P V \right) &= \frac{\Vert\mathbf{q}\Vert^3}{2 \pi \Lambda^2 m_V^2 m_{\tilde{P}}} m_P \left(\Lambda \left(g_{0,1}-g_{0,2}\right)+\left(g_{1,4}-g_{1,3}\right) \sqrt{m_V^2+\Vert\mathbf{q}\Vert^2}\right)^2 \, , \\
  \Gamma\left( \tilde{P} \to P^* V \right) &= \frac{m_{P^*} \Vert \mathbf{q} \Vert^3}{\pi  \Lambda ^2 m_{\tilde{P}}} g_{1,6}^2 \, , \\
  \Gamma\left( \tilde{P}^* \to P V \right) &= \frac{m_P \Vert \mathbf{q} \Vert^3}{3 \pi  \Lambda ^2 m_{\tilde{P}^*}} g_{1,6}^2 \, ,
\end{align}
\begin{multline}
  \Gamma\left( \tilde{P}^* \to P^* V \right) = \frac{\Vert\mathbf{q}\Vert^3 m_{P^*}}{6 \pi \Lambda^2 m_V^2 m_{\tilde{P}^*}} \Bigg(3 \Lambda \left(g_{0,1}-g_{0,2}\right) \left(\Lambda \left( g_{0,1}-g_{0,2}\right)+2 \left(g_{1,4}-g_{1,3}\right) \sqrt{m_V^2+\Vert\mathbf{q}\Vert^2}\right) \\
  +\left(3 \left(g_{1,3}-g_{1,4}\right){}^2+4 g_{1,6}^2\right) m_V^2+3 \Vert\mathbf{q}\Vert^2 \left(g_{1,3}-g_{1,4}\right)^2\Bigg)\, .
\end{multline}
Benché le loro espressioni non si siano molto semplificate, da queste ultime è possibile fare una lettura più immediata della situazione. Anzitutto gli scalari non contribuiscono (come atteso). Invece, le strutture simmetriche in $v \gamma$ e $v v$ danno gli stessi contributi e, tra quelle antisimmetriche, quella mista in $v \gamma$ non contribuisce, mentre quella in $\gamma \gamma$ è la sola responsabile dei decadimenti $\Gamma\left( \tilde{P} \to P^* V \right)$ e $\Gamma\left( \tilde{P}^* \to P V \right)$. Infine, si nota l'esistenza di un termine di interferenza tra le strutture simmetriche e quella antisimmetrica in $\gamma \gamma$ nel decadimento $\Gamma\left( \tilde{P}^* \to P^* V \right)$.

Tuttavia si osserva che questo tipo di analisi è via via più difficile per strutture più complesse, dovendo considerare tensori $\Gamma$ con sempre più indici e ridurre le diverse combinazioni (corrispondenti alle diverse possibili sostituzioni di $v$ e $\gamma$) al numero minimo di oggetti distinti con proprietà tensoriali definite. Pertanto, per valori di $n$ e $m$ maggiori, è probabilmente più fattibile considerare semplicemente le possibili stringhe di $v$ e $\gamma$.
%Riporto di seguito le possibili strutture per
%\begin{equation}
%  \mathcal{G}_{\mu_1 \dots \mu_n} =  \sum_{m,k} \frac{g_{m,k}}{\Lambda^m} \; \Gamma\indices{^{(k)}_{\mu_1 \dots \mu_n}^{\nu_1 \dots \nu_m \alpha}} i \partial_{\nu_1} \dots i \partial_{\nu_m} \rho_\alpha \, ,
%\end{equation}
%fino al valore minimo di $m$ considerato (si sottintende che le costanti di accoppiamento siano diverse per ogni doppietto) e le corrispondenti larghezze di decadimento, dove $q$ è l'impulso del vettore leggero nello stato finale.
%
%\paragraph{Decadimenti di $\tilde{H}$}
%\begin{equation}
%  \mathcal{G} = g_0 \slashed{\rho} + \frac{g_1}{\Lambda} i \slashed{\partial}\slashed{\rho} \, ,
%\end{equation}
%
%\begin{align}
%  \Gamma\left( \tilde{P} \to P V \right) &= \frac{m_P \Vert \mathbf{q} \Vert^3}{2 \pi m_V^2 m_{\tilde{P}}} g_0^2 \, , \\
%  \Gamma\left( \tilde{P} \to P^* V \right) &= \frac{m_{P^*} \Vert \mathbf{q} \Vert^3}{\pi  \Lambda ^2 m_{\tilde{P}}} g_1^2 \, , \\
%  \Gamma\left( \tilde{P}^* \to P V \right) &= \frac{m_P \Vert \mathbf{q} \Vert^3}{3 \pi  \Lambda ^2 m_{\tilde{P}^*}} g_1^2 \, , \\
%  \Gamma\left( \tilde{P}^* \to P^* V \right) &= \frac{m_{P^*} \Vert \mathbf{q} \Vert^3}{6 \pi  \Lambda ^2 m_V^2 m_{\tilde{P}^*}} \left(3 g_0^2 \Lambda ^2+4 g_1^2 m_V^2\right) \, .
%\end{align}
%
%\paragraph{Decadimenti di $S$}
%\begin{equation}
%  \mathcal{G} = g_0 \slashed{\rho} \, , 
%\end{equation}
%
%\begin{align}
%  \Gamma\left( P^*_0 \to P^* V \right) =& \frac{m_{P^*} \Vert \mathbf{q} \Vert}{2 \pi  m_{P_0^*} m_V^2} g_0^2 \left(3 m_V^2+ \Vert \mathbf{q} \Vert^2\right) \, , \\
%  \Gamma\left( P'_1 \to P V \right) =& \frac{m_P \Vert \mathbf{q} \Vert}{6 \pi  m_V^2 m_{P'_1}} g_0^2 \left(3 m_V^2+ \Vert \mathbf{q} \Vert^2\right) \, , \\
%  \Gamma\left( P'_1 \to P^* V \right) =& \frac{m_{P^*} \Vert \mathbf{q} \Vert }{3 \pi  m_V^2 m_{P'_1}} g_0^2 \left(3 m_V^2+ \Vert \mathbf{q} \Vert^2\right) \, . 
%\end{align}
%
%\paragraph{Decadimenti di $T^\mu$}
%\begin{equation}
%  \mathcal{G}_\mu = g_0 \rho_\mu + \frac{g_{1,1}}{\Lambda} i \slashed{\partial} \rho_\mu + \frac{g_{1,2}}{\Lambda} i \partial_\mu \slashed{\rho} + \frac{g_{2,1}}{\Lambda^2} \left( i \partial \right)^2 \rho_\mu + \frac{g_{2,2}}{\Lambda^2} \frac{1}{2} \left( i \partial_\mu i \slashed{\partial} + i \slashed{\partial} i \partial_\mu \right) \slashed{\rho} \, ,
%\end{equation}
%
%\begin{multline}
%  \Gamma\left( P_1 \to P V \right) = \frac{\Vert \mathbf{q} \Vert m_P }{18 \pi  \Lambda ^4 m_{P_1} m_V^2}
%  \Bigg( (2 \Lambda ^2 g_{1,1}^2 \left(m_V^2+\Vert \mathbf{q} \Vert^2\right) \left(3 m_V^2+\Vert \mathbf{q} \Vert^2\right) \\
%  -4 g_0 \Lambda ^2 \left(\Lambda  \sqrt{m_V^2+\Vert \mathbf{q} \Vert^2} \left(3 g_{1,1} m_V^2+\Vert \mathbf{q} \Vert^2 \left(g_{1,1}+g_{1,2}\right)\right)-g_{2,1} m_V^2 \left(3 m_V^2+\Vert \mathbf{q} \Vert^2\right)+\Vert \mathbf{q} \Vert^2 \left(-g_{2,2}\right) m_V^2\right) \\
% +\Vert \mathbf{q} \Vert^4 \left(2 \Lambda ^2 g_{1,2}^2+g_{2,2}^2 m_V^2\right)-4 \Lambda  \Vert \mathbf{q} \Vert^2 g_{1,2} g_{2,1} m_V^2 \sqrt{m_V^2+\Vert \mathbf{q} \Vert^2} \\
% +4 \Lambda  g_{1,1} \left(\Lambda  \Vert \mathbf{q} \Vert^2 g_{1,2} \left(m_V^2+\Vert \mathbf{q} \Vert^2\right)-m_V^2 \sqrt{m_V^2+\Vert \mathbf{q} \Vert^2} \left(g_{2,1} \left(3 m_V^2+\Vert \mathbf{q} \Vert^2\right)+\Vert \mathbf{q} \Vert^2 g_{2,2}\right)\right) \\
% +2 g_{2,1}^2 m_V^4 \left(3 m_V^2+\Vert \mathbf{q} \Vert^2\right)+4 \Vert \mathbf{q} \Vert^2 g_{2,1} g_{2,2} m_V^4+2 g_0^2 \Lambda ^4 \left(3 m_V^2+\Vert \mathbf{q} \Vert^2\right)\Bigg) \, ,
%\end{multline}
%
%\begin{multline}
%  \Gamma\left( P_1 \to P^* V \right) = \frac{\Vert \mathbf{q} \Vert m_{P^*}}{18 \pi  \Lambda ^4 m_{P_1} m_V^2}
%  \Bigg(\Lambda ^2 g_{1,1}^2 \left(m_V^2+\Vert \mathbf{q} \Vert^2\right) \left(3 m_V^2+\Vert \mathbf{q} \Vert^2\right) \\
%  +2 g_0 \Lambda ^2 \left(-\Lambda  \sqrt{m_V^2+\Vert \mathbf{q} \Vert^2} \left(3 g_{1,1} m_V^2+\Vert \mathbf{q} \Vert^2 \left(g_{1,1}+g_{1,2}\right)\right)+g_{2,1} m_V^2 \left(3 m_V^2+\Vert \mathbf{q} \Vert^2\right)+\Vert \mathbf{q} \Vert^2 g_{2,2} m_V^2\right) \\
%  +\Vert \mathbf{q} \Vert^4 \left(\Lambda ^2 g_{1,2}^2+5 g_{2,2}^2 m_V^2\right)-2 \Lambda  \Vert \mathbf{q} \Vert^2 g_{1,2} g_{2,1} m_V^2 \sqrt{m_V^2+\Vert \mathbf{q} \Vert^2} \\
%  +2 \Lambda  g_{1,1} \left(\Lambda  \Vert \mathbf{q} \Vert^2 g_{1,2} \left(m_V^2+\Vert \mathbf{q} \Vert^2\right)-m_V^2 \sqrt{m_V^2+\Vert \mathbf{q} \Vert^2} \left(g_{2,1} \left(3 m_V^2+\Vert \mathbf{q} \Vert^2\right)+\Vert \mathbf{q} \Vert^2 g_{2,2}\right)\right) \\
%  +g_{2,1}^2 m_V^4 \left(3 m_V^2+\Vert \mathbf{q} \Vert^2\right)+2 \Vert \mathbf{q} \Vert^2 g_{2,1} g_{2,2} m_V^4+g_0^2 \Lambda ^4 \left(3 m_V^2+\Vert \mathbf{q} \Vert^2\right)\Bigg) \, ,
%\end{multline}
%
%\begin{equation}
%  \Gamma\left( P^*_2 \to P V \right) = \frac{\Vert \mathbf{q} \Vert^5 m_P}{10 \pi  \Lambda ^4 m_{P_2^*}} g_{2,2}^2 \, ,
%\end{equation}
%
%\begin{multline}
%  \Gamma\left( P^*_2 \to P^* V \right) = \frac{\Vert \mathbf{q} \Vert m_{P^*}}{30 \pi  \Lambda ^4 m_{P_2^*} m_V^2}
%  \Bigg(5 \Lambda ^2 g_{1,1}^2 \left(m_V^2+\Vert \mathbf{q} \Vert^2\right) \left(3 m_V^2+\Vert \mathbf{q} \Vert^2\right) \\
%    -10 g_0 \Lambda ^2 \left(\Lambda  \sqrt{m_V^2+\Vert \mathbf{q} \Vert^2} \left(3 g_{1,1} m_V^2+\Vert \mathbf{q} \Vert^2 \left(g_{1,1}+g_{1,2}\right)\right)-g_{2,1} m_V^2 \left(3 m_V^2+\Vert \mathbf{q} \Vert^2\right)+\Vert \mathbf{q} \Vert^2 \left(-g_{2,2}\right) m_V^2\right) \\
%  +\Vert \mathbf{q} \Vert^4 \left(5 \Lambda ^2 g_{1,2}^2+7 g_{2,2}^2 m_V^2\right)-10 \Lambda  \Vert \mathbf{q} \Vert^2 g_{1,2} g_{2,1} m_V^2 \sqrt{m_V^2+\Vert \mathbf{q} \Vert^2} \\
%  +10 \Lambda  g_{1,1} \left(\Lambda  \Vert \mathbf{q} \Vert^2 g_{1,2} \left(m_V^2+\Vert \mathbf{q} \Vert^2\right)-m_V^2 \sqrt{m_V^2+\Vert \mathbf{q} \Vert^2} \left(g_{2,1} \left(3 m_V^2+\Vert \mathbf{q} \Vert^2\right)+\Vert \mathbf{q} \Vert^2 g_{2,2}\right)\right) \\
%+5 g_{2,1}^2 m_V^4 \left(3 m_V^2+\Vert \mathbf{q} \Vert^2\right)+10 \Vert \mathbf{q} \Vert^2 g_{2,1} g_{2,2} m_V^4+5 g_0^2 \Lambda ^4 \left(3 m_V^2+\Vert \mathbf{q} \Vert^2\right)\Bigg) \, .
%\end{multline}
%
%\paragraph{Decadimenti di $X^\mu$}
%\begin{equation}
%  \mathcal{G}_\mu = g_0 \rho_\mu + \frac{g_{1,1}}{\Lambda} i \slashed{\partial} \rho_\mu + \frac{g_{1,2}}{\Lambda} i \partial_\mu \slashed{\rho} \, ,
%\end{equation}
%
%\begin{align}
%  \Gamma\left( P^*_1 \to P V \right) &= \frac{\Vert \mathbf{q} \Vert^3 \left(g_{1,1}-g_{1,2}\right)^2 m_P}{18 \pi  \Lambda ^2 m_{P_1^*}} \, , \\
%  \Gamma\left( P^*_1 \to P^* V \right) &= \frac{\Vert \mathbf{q} \Vert^3 m_{P^*}}{18 \pi  \Lambda^2 m_{P_1^*} m_V^2} \left(2 \left(4 g_{1,1}^2+7 g_{1,2} g_{1,1}+4 g_{1,2}^2\right) m_V^2+3 \Vert \mathbf{q} \Vert^2 \left(g_{1,1}+g_{1,2}\right)^2\right) \, , \\
%  \Gamma\left( P_2 \to P V \right) &= \frac{\Vert\mathbf{q}\Vert^3 m_P}{30 \pi \Lambda ^2 m_{P_2} m_V^2} \left(g_{1,1}+g_{1,2}\right)^2 \left(5 m_V^2+2 \Vert\mathbf{q}\Vert^2\right)  \, , \\
%  \Gamma\left( P_2 \to P^* V \right) &= \frac{\Vert\mathbf{q}\Vert^3 m_{P^*} }{30 \pi \Lambda^2 m_{P_2} m_V^2} \left(10 \left(g_{1,1}^2+g_{1,2} g_{1,1}+g_{1,2}^2\right) m_V^2+3 \Vert\mathbf{q}\Vert^2 \left(g_{1,1}+g_{1,2}\right)^2\right) \, . 
%\end{align}
%
%\paragraph{Decadimenti di $X'^{\mu \nu}$}
%\begin{multline}
%  \mathcal{G}_{\mu \nu} = \frac{g_1}{\Lambda} \frac{1}{2} \left( i \partial_\mu \rho_\nu + \mu \leftrightarrow \nu \right) + \frac{g_{2,1}}{\Lambda^2} \frac{1}{2} \left( i \partial_\mu i \partial_\nu + \mu \leftrightarrow \nu \right) \slashed{\rho} + \frac{g_{2,2}}{\Lambda^2} \frac{1}{4} \left( i \slashed{\partial} i \partial_\mu \rho_\nu + i \partial_\mu i \slashed{\partial} \rho_\nu + \mu \leftrightarrow \nu \right) \\ + \frac{g_{3,1}}{\Lambda^3} \frac{1}{2} \left(i \partial \right)^2 \left( i \partial_\mu \rho_\nu + \mu \leftrightarrow \nu \right) + \frac{g_{3,2}}{\Lambda^3} \frac{1}{6} \left( i \slashed{\partial} i \partial_\mu i \partial_\nu + i \partial_\mu i \slashed{\partial} i \partial_\nu + i \partial_\mu i \partial_\nu i \slashed{\partial}  + \mu \leftrightarrow \nu \right) \slashed{\rho} \, ,
%\end{multline}
%
%\begin{multline}
%  \Gamma\left( P'_2 \to P V \right) = \frac{\Vert\mathbf{q}\Vert^3 m_P}{150 \pi  \Lambda ^6 m_V^2 m_{P'_2}} \Bigg(3 \Lambda ^2 g_{2,2}^2 \left(m_V^2+\Vert\mathbf{q}\Vert^2\right) \left(5 m_V^2+2 \Vert\mathbf{q}\Vert^2\right) \\
%    -6 g_1 \Lambda^2 \left(\Lambda \sqrt{m_V^2+\Vert\mathbf{q}\Vert^2} \left(5 g_{2,2} m_V^2+2 \Vert\mathbf{q}\Vert^2 \left(g_{2,1}+g_{2,2}\right)\right)-2 \Vert\mathbf{q}\Vert^2 g_{3,2} m_V^2-g_{3,1} \left(2 \Vert\mathbf{q}\Vert^2 m_V^2+5 m_V^4\right)\right) \\
%    +2 \Vert\mathbf{q}\Vert^4 \left(3 \Lambda ^2 g_{2,1}^2+2 g_{3,2}^2 m_V^2\right)-12 \Lambda  \Vert\mathbf{q}\Vert^2 g_{2,1} g_{3,1} m_V^2 \sqrt{m_V^2+\Vert\mathbf{q}\Vert^2} \\
%  +6 \Lambda g_{2,2} \left(2 \Lambda  \Vert\mathbf{q}\Vert^2 g_{2,1} \left(m_V^2+\Vert\mathbf{q}\Vert^2\right)-m_V^2 \sqrt{m_V^2+\Vert\mathbf{q}\Vert^2} \left(5 g_{3,1} m_V^2+2 \Vert\mathbf{q}\Vert^2 \left(g_{3,1}+g_{3,2}\right)\right)\right) \\
%+3 g_{3,1}^2 m_V^4 \left(5 m_V^2+2 \Vert\mathbf{q}\Vert^2\right)+12 \Vert\mathbf{q}\Vert^2 g_{3,1} g_{3,2} m_V^4+3 g_1^2 \Lambda ^4 \left(5 m_V^2+2 \Vert\mathbf{q}\Vert^2\right)\Bigg) \, ,
%\end{multline}
%
%\begin{multline}
%  \Gamma\left( P'_2 \to P^* V \right) = \frac{\Vert\mathbf{q}\Vert^3 m_{P^*}}{75 \pi  \Lambda ^6 m_V^2 m_{P'_2}} \Bigg(\Lambda ^2 g_{2,2}^2 \left(m_V^2+\Vert\mathbf{q}\Vert^2\right) \left(5 m_V^2+2 \Vert\mathbf{q}\Vert^2\right) \\
%  +2 g_1 \Lambda ^2 \left(-\Lambda  \sqrt{m_V^2+\Vert\mathbf{q}\Vert^2} \left(5 g_{2,2} m_V^2+2 \Vert\mathbf{q}\Vert^2 \left(g_{2,1}+g_{2,2}\right)\right)+2 \Vert\mathbf{q}\Vert^2 g_{3,2} m_V^2+g_{3,1} \left(2 \Vert\mathbf{q}\Vert^2 m_V^2+5 m_V^4\right)\right) \\
%  +2 \Vert\mathbf{q}\Vert^4 \left(\Lambda ^2 g_{2,1}^2+4 g_{3,2}^2 m_V^2\right)-4 \Lambda  \Vert\mathbf{q}\Vert^2 g_{2,1} g_{3,1} m_V^2 \sqrt{m_V^2+\Vert\mathbf{q}\Vert^2} \\
%  +2 \Lambda  g_{2,2} \left(2 \Lambda  \Vert\mathbf{q}\Vert^2 g_{2,1} \left(m_V^2+\Vert\mathbf{q}\Vert^2\right)-m_V^2 \sqrt{m_V^2+\Vert\mathbf{q}\Vert^2} \left(5 g_{3,1} m_V^2+2 \Vert\mathbf{q}\Vert^2 \left(g_{3,1}+g_{3,2}\right)\right)\right) \\
%  +g_{3,1}^2 m_V^4 \left(5 m_V^2+2 \Vert\mathbf{q}\Vert^2\right)+4 \Vert\mathbf{q}\Vert^2 g_{3,1} g_{3,2} m_V^4+g_1^2 \Lambda ^4 \left(5 m_V^2+2 \Vert\mathbf{q}\Vert^2\right)\Bigg) \, ,
%\end{multline}
%
%\begin{equation}
%  \Gamma\left( P^*_3 \to P V \right) = \frac{4 \Vert \mathbf{q} \Vert^7 m_P}{105 \pi \Lambda^6 m_{P_3^*}} g_{3,2}^2 \, ,
%\end{equation}
%
%\begin{multline}
%  \Gamma\left( P^*_3 \to P^* V \right) = \frac{\Vert\mathbf{q}\Vert^3 m_{P^*}}{210 \pi  \Lambda ^6 m_{P_3^*} m_V^2} \Bigg(7 \Lambda ^2 g_{2,2}^2 \left(m_V^2+\Vert\mathbf{q}\Vert^2\right) \left(5 m_V^2+2 \Vert\mathbf{q}\Vert^2\right) \\
%    -14 g_1 \Lambda ^2 \left(\Lambda  \sqrt{m_V^2+\Vert\mathbf{q}\Vert^2} \left(5 g_{2,2} m_V^2+2 \Vert\mathbf{q}\Vert^2 \left(g_{2,1}+g_{2,2}\right)\right)-2 \Vert\mathbf{q}\Vert^2 g_{3,2} m_V^2-g_{3,1} \left(2 \Vert\mathbf{q}\Vert^2 m_V^2+5 m_V^4\right)\right) \\
%  +2 \Vert\mathbf{q}\Vert^4 \left(7 \Lambda ^2 g_{2,1}^2+10 g_{3,2}^2 m_V^2\right)-28 \Lambda  \Vert\mathbf{q}\Vert^2 g_{2,1} g_{3,1} m_V^2 \sqrt{m_V^2+\Vert\mathbf{q}\Vert^2} \\
%  +14 \Lambda  g_{2,2} \left(2 \Lambda  \Vert\mathbf{q}\Vert^2 g_{2,1} \left(m_V^2+\Vert\mathbf{q}\Vert^2\right)-m_V^2 \sqrt{m_V^2+\Vert\mathbf{q}\Vert^2} \left(5 g_{3,1} m_V^2+2 \Vert\mathbf{q}\Vert^2 \left(g_{3,1}+g_{3,2}\right)\right)\right)\\ 
%  +7 g_{3,1}^2 m_V^4 \left(5 m_V^2+2 \Vert\mathbf{q}\Vert^2\right)+28 \Vert\mathbf{q}\Vert^2 g_{3,1} g_{3,2} m_V^4+7 g_1^2 \Lambda ^4 \left(5 m_V^2+2 \Vert\mathbf{q}\Vert^2\right)\Bigg) \, .
%\end{multline}
%
%\paragraph{Decadimenti di $F^{\mu \nu}$}
%\begin{equation}
%  \mathcal{G}_{\mu \nu} = \frac{g_1}{\Lambda} \frac{1}{2} \left( i \partial_\mu \rho_\nu + \mu \leftrightarrow \nu \right) + \frac{g_{2,1}}{\Lambda^2} \frac{1}{2} \left( i \partial_\mu i \partial_\nu + \mu \leftrightarrow \nu \right) \slashed{\rho} + \frac{g_{2,2}}{\Lambda^2} \frac{1}{4} \left( i \slashed{\partial} i \partial_\mu \rho_\nu + i \partial_\mu i \slashed{\partial} \rho_\nu + \mu \leftrightarrow \nu \right) \, ,
%\end{equation}
%
%\begin{align}
%  \Gamma\left( P^{\prime *}_2 \to P V \right) &= \frac{\Vert\mathbf{q}\Vert^5 m_P}{150 \pi \Lambda^4 m_{P^{\prime *}_2}} \left(g_{2,2}-2 g_{2,1}\right)^2 \, , \\
%  \Gamma\left( P^{\prime *}_2 \to P^* V \right) &= \frac{\Vert\mathbf{q}\Vert^5 m_{P^*}}{75 \pi \Lambda^4 m_V^2 m_{P^{\prime *}_2}} \left(\left(13 g_{2,1}^2+22 g_{2,2} g_{2,1}+12 g_{2,2}^2\right) m_V^2+5 \Vert\mathbf{q}\Vert^2 \left(g_{2,1}+g_{2,2}\right)^2\right)\, , \\
%  \Gamma\left( P_3 \to P V \right) &= \frac{\Vert\mathbf{q}\Vert^5 m_P }{105 \pi \Lambda ^4 m_{P_3} m_V^2} \left(g_{2,1}+g_{2,2}\right)^2 \left(7 m_V^2+3 \Vert\mathbf{q}\Vert^2\right)\, , \\
%  \Gamma\left( P_3 \to P^* V \right) &=  \frac{\Vert\mathbf{q}\Vert^5 m_{P^*}}{210 \pi  \Lambda ^4 m_{P_3} m_V^2} \left(7 \left(4 g_{2,1}^2+4 g_{2,2} g_{2,1}+3 g_{2,2}^2\right) m_V^2+8 \Vert\mathbf{q}\Vert^2 \left(g_{2,1}+g_{2,2}\right)^2\right)\, . 
%\end{align}

\begin{thebibliography}{}
  \bibitem{book:Greiner}
  Greiner, Walter, and Joachim Reinhardt. \textit{Field quantization.} Springer Science \& Business Media, 2013.
  \bibitem{article:Tyutin}
  Tyutin, I. V., and B. B. Lokhvitskii. "Charge conjugation of non-Abelian gauge fields." \textit{Soviet Physics Journal} 25.4 (1982): 346-348.
  \bibitem{article:Luke_Manohar}
  Luke, M. E., and Manohar, A.V. "Reparametrization Invariance Constraints on Heavy Particle Effective Field Theories." \textit{Phys. Lett. B} 286 (1992): 348.
  \bibitem{article:Georgi}
  Georgi, Howard. "An effective field theory for heavy quarks at low energies." \textit{Physics Letters B} 240.3-4 (1990): 447-450.
  \bibitem{article:Casalbuoni}
  Casalbuoni, Roberto, et al. "Light vector resonances in the effective chiral Lagrangian for heavy mesons." \textit{Physics Letters B} 292.3-4 (1992): 371-376.
\end{thebibliography}

\end{document}
